\chapter{总结与展望}
\label{chap:conclusion}

\section{工作总结}
\label{sec
}

针对空天目标RATR技术在小样本条件下,因数据稀缺性与雷达信号固有复杂相互作用而导致的深度模型性能受限这一核心瓶颈,本文以元学习为主要理论框架,聚焦于HRRP数据,开展了系统性的研究工作。研究旨在通过改进元学习机制,使其更好地适应HRRP数据的特性,从而提升RATR系统在复杂、动态、数据受限真实环境下的识别性能与适应能力。主要研究内容与成果如下:

\textbf{1. 基于动态图元学习的噪声环境下小样本HRRP识别方法}

认识到信号质量是特征提取的基础,第三章研究工作从应对普遍存在的噪声干扰入手。在小样本条件下,噪声对识别性能的劣化尤为显著。为此,本文提出了一种基于动态图元学习的鲁棒识别方法HRRPGraphNet++。该方法将HRRP样本建模为图结构,并设计了融合物理先验与数据驱动注意力的混合动态图构建策略,旨在自适应地捕捉噪声影响下的样本内部关联。结合GNN进行特征提取,并将此图学习模块嵌入改进的MAML++元学习框架中。通过元训练学习在少量含噪样本下快速适应的策略,实验结果显示,该方法相比基线方法在低信噪比下展现出更好的性能稳定性,为后续处理更复杂的HRRP特性提供了基础。

\textbf{2. 基于样本间关系挖掘的跨角度小样本HRRP元学习识别方法}

第四章进一步聚焦于HRRP识别中更为核心的物理瓶颈——角度敏感性。角度敏感性导致巨大的类内差异,严重挑战了小样本学习的假设。为克服此困难,本文提出了一种基于样本间关系挖掘的元学习方法GAF-MLGNN。该方法不再视样本为孤立点,而是利用GAF编码样本内信息,并构建任务图,通过GNN显式地建模和挖掘样本间的关系。其中,边权由可学习的MLP动态计算,旨在捕捉超越简单相似性的复杂关联。将图学习模块嵌入MLGNN元学习框架,通过在角度多样性任务上训练,学习适应角度变化的元知识。实验结果表明,该方法在标准小样本设置下性能优越,并通过挖掘样本间关系,展现了提升跨角度泛化能力的潜力,这标志着在处理HRRP固有复杂性上迈进了一步。

\textbf{3. 提出了基于跨模态语义嵌入的小样本HRRP元学习识别方法}

认识到仅依赖物理特征在小样本和细粒度区分场景下的局限性,第五章研究探索了引入外部语义信息的可能性。当物理特征相似或受扰严重时,语义信息能提供重要的补充判别线索。为解决一维HRRP与主流VLM的模态鸿沟问题,本文提出了协同跨模态适配方法SHARP。通过设计可训练的1D-to-2D适配器,利用冻结的VLM视觉编码器提取特征,并采用协同训练策略缓解“语义偏见”,确保提取的特征既有语义相关性又保留HRRP的判别细节。在小样本识别阶段,结合预训练的SemAlign模块进一步利用语义精炼原型。实验结果显示,该方法能有效迁移VLM知识,显著提升了小样本HRRP识别精度,尤其是在区分相似目标方面,为信息融合提供了新的视角。

综上所述,本文围绕小样本HRRP识别的核心挑战,首先着眼于提升模型在噪声环境下的基础鲁棒性,随后聚焦于克服HRRP固有的角度敏感性难题,最终探索了融合外部语义知识以增强小样本识别效果的途径,提出了一系列基于元学习的改进方法。通过仿真和部分实测数据实验,验证了所提方法在一定程度上缓解了噪声影响、角度敏感性问题,并探索了语义信息利用的潜力,为提升复杂环境下RATR系统的性能提供了若干可行的思路和技术途径。

\section{下一步工作展望}
\label{sec}

尽管本文在基于元学习的小样本HRRP识别方面进行了一些探索并取得初步进展,但研究仍存在不足,且该领域尚有广阔空间值得未来深入研究。结合本文工作,后续工作研究可从以下几个方面进一步展开:

\textbf{一是面向实测场景的领域自适应元学习方面,} 将小样本识别方法应用于实测场景面临更复杂噪声、杂波、目标多样性以及不同雷达、环境条件带来的显著“域差异”,提升算法在真实部署中的鲁棒性和泛化能力至关重要。因此,深入研究如何将元学习与领域自适应技术如分布对齐、对抗学习、模型适应策略等深度融合,以克服域差异,实现跨域场景下的高效小样本识别,是提升算法实用价值的关键。

\textbf{二是面向端侧部署的大模型能力迁移与轻量化方面,} 尽管利用大规模预训练模型VLM、LLM等能显著提升性能,但其巨大的计算与存储需求阻碍了在导弹导引头等资源极端受限平台上的部署。因此,研究如何通过知识蒸馏、模型剪枝、量化及轻量化网络设计等技术,将大模型的强大泛化能力与细粒度识别知识,高效迁移压缩至一个能直接处理HRRP信号的紧凑模型中。这对于在满足端侧实时性与低功耗要求的同时,保持小样本识别的高性能至关重要。

\textbf{三是结合持续学习与开集识别的动态适应能力方面, }真实的识别环境目标类型和环境特性也随时间动态演变,易出现训练阶段未知目标或干扰。将小样本学习能力与持续学习、开集识别等技术相结合,研究如何使模型在保持对旧知识记忆的同时,能够增量式地适配新出现的目标类别或环境变化,并具备识别未知目标的能力,是赋予HRRP识别系统真实动态环境适应性的重要研究方向。