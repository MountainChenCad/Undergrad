
%%% Local Variables:
%%% mode: latex
%%% TeX-master: "../main"
%%% End:

\begin{ack}
报考国防科大,始于少年时的荧屏记忆。哈军工前辈的身影,学者风骨,军人本色,在那个年代交织辉映。曾向往那样的场景:与战友并肩奔跑,也探讨着科学;灯下捧着饭盒,仍调试着仪器。那是心之所向。如今在科大,昔日憧憬已是日常,前行的脚步也因此更加踏实。

深深感谢国防科技大学这片沃土。常想,若非在此经历“两个转变”的心志淬炼,我难有今日这份笃定。是科大,推着我一步步走出舒适区,去迎接那些曾经的畏惧:从克服内心障碍、站上英语演讲台获得认可,到连续两年带队远赴巴黎、在国际赛场上历练,终获大型赛事奖项十余项。这些昔日未敢奢望的突破,都源于科大提供的平台与机遇。这片独特的熔炉,锻造了勇气,也见证了成长。

在此,谨向我的导师户盼鹤副教授致以最深的敬意与感谢。读研路上,户老师是引路人,亦是同行者。他将宝贵的经验倾囊相授,以恳切的教诲照亮前路。多少个夜晚,无论多晚,他总把学生的事放在心上,耐心解答我的每一个疑问。从课题方向的把握到技术细节的斟酌,户老师总是与我平等交流,细致探讨,既有学者的严谨,亦有师长的温和。他常教导我们,要有纯正的学术追求,要从部队的实际出发,去解决真问题。这份言传身教,如春风化雨,将伴我一生。

感谢我的招生老师游鹏老师。您是我初入科大的领路人。正是您的关心与指导,让我得以更快地融入这里的学习与生活节奏,开始认识并实现自我价值。

感谢教研室的黎湘老师、刘永祥老师、姜卫东老师、高勋章老师、刘振老师、刘丽老师、张双辉老师、刘天鹏老师、张文鹏老师、杨威老师、沈亲沐老师、卢哲俊老师、夏靖远老师、苏晓龙老师、智帅峰老师、张新禹老师、李俊老师、刘烁伟老师、李铭洋老师、林煌星老师、张志远老师和李瑞泽老师。感谢您们为我的课题严格把关,提出的宝贵意见如同路标,指引方向。亦感谢您们在日常中给予的关心与帮助。感谢办公室助理张煌、钟仪、颜学钦、王肖、高柯,感谢硬件工程师苏建伟、杨近松、朱乾坤。感谢你们的默默付出与支持,让我得以更加心无旁骛地投入到学习与研究之中。

感谢刘旗师兄、邱祥风师兄、程耘师兄、姜辣师兄、潘之梁师兄、郭金兴师兄、廖淮璋师兄、杨志雄师兄、李伟杰师兄、周洁师姐、刘阿飞师兄、张家伟师兄、易拓源师兄、卞小贝师姐、张青青师姐、王朋朋师兄、孙潇师兄、陈冠潮师兄、孙浩彬师兄、王泽昊师兄、张振家师兄、程晨师兄、涂晓未师兄。与你们的交流,常能激荡出思想的火花。无论学业难题还是生活点滴,你们总能凭借过来人的经验,给予我中肯而温暖的建议。这种亦师亦友的情谊,弥足珍贵。

感谢湖南省科学技术厅的经费支持。主持湖南省自然科学基金青年学生基础研究项目的这段经历,不仅是科研能力的集中锻炼,更是对独立承担责任、系统推进工作的宝贵实践。这段经历为我未来的科研道路奠定了坚实基础,其益无穷。

感谢理学院生化系Zhu's Lab、计算机学院天河生物医药团队、前沿交叉学院光电所、军政基础教育学院军事外语系、理学院数学系我的指导老师们,你们是我科研生涯的启蒙者。在全球、全国顶级赛事四年的锻炼,极大拓展了我的学科视野,夯实了我的科研基本功,赋予了我独立承担课题、勇于探索创新的底气。

感谢电子科学学院一大队一队全体人员,感谢大家四年的陪伴。感谢贺海军、高兴亮、黄源、叶俊、杨亚洲、辛熙鹤、李威、徐佳、邹斐帆队长和教导员们,你们是我军旅人生的灯塔,没有你们搭的“梯子”,我也无法吃上“盘子”。感谢同级的范文刚、徐丞柏、黄意扬、韩天放、刘洋、奚齐、周骜、徐浚熙、杨颖、黄昕睿、陆嘉杰、范昊、冯世宇、杨姿等战友,我们同甘共苦,希望你们未来越来越好!

最后,感谢我的家人。你们默默的理解与无条件的支持,是我能够安心求学、勇往直前的最坚实后盾和最温暖港湾。

行文至此,思绪万千,谨以此四言小诗作结,聊表寸心:

\begin{center}
负笈湘皋,问道穷源。\\
格物致知,淬砺心坚。\\
家国之任,未敢等闲。\\
来日骋志,更谱新篇。\\
鹏程初展,好风正悬。\\
前路修远,步履弥坚。\\
感念师友,德谊绵长。\\
丹心如铁,辉映朝阳。
\end{center}

\vspace{2\baselineskip} % 增加一些垂直间距
\begin{flushright}
{\kaishu 陈凌峰~~~~~~~~~~~} \\ % 请替换为您的姓名
2025年5月于长沙 % 请替换为实际年月
\end{flushright}
  
\end{ack}
