% TeX
\chapter{大语言模型生成的电磁语义描述}
\label{chap:llmsem}
% --- Claude-3.7 Descriptions ---
\section{Claude-3.7-Sonnet 生成的描述}
\label{sec:claude_desc}
\footnotesize
{\begin{description} % Using description list for clear labeling
    \item[\textbf{EP-3E}]
    {The EP-3E Aries II presents a highly distinctive HRRP signature characterized by its large, elongated fuselage and high-wing turboprop configuration. The four wing-mounted Allison T56 turboprop engines create prominent scattering centers with their nacelles and propeller assemblies generating complex dynamic returns. The aircraft's numerous external intelligence-gathering antennas, radomes, and sensor pods distributed across the airframe produce a rich, textured HRRP with multiple secondary scatterers superimposed on the main fuselage return. The high-mounted straight wings form distinctive corner reflectors at their junction with the fuselage, while the large vertical stabilizer and horizontal tailplane create additional strong returns at their respective positions. The glazed nose section housing the flight deck and mission systems operators produces a characteristic rounded leading edge return that differentiates it from transport aircraft with more angular nose profiles. The unique 'canoe' fairings beneath the fuselage housing specialized SIGINT equipment further complicate the profile, making the EP-3E readily distinguishable from its P-3 Orion ancestor and other maritime patrol platforms.}

    \item[\textbf{F18}] % Assuming F18 refers to Super Hornet based on description
    The F/A-18E/F Super Hornet's HRRP signature is characterized by strong returns from the twin engine inlets positioned below the prominent leading-edge extensions (LEX), forming distinct scattering centers. The twin, canted vertical stabilizers create characteristic dual returns, especially at oblique aspect angles, differentiating it from single-tail fighters. Complex corner reflections occur at the wing-fuselage junction points, while edge diffraction from the moderately swept wings and horizontal stabilizers contribute additional peaks to the range profile. The aircraft's relatively blended wing-body design moderates some returns, but it lacks true stealth features. The Super Hornet's larger airframe compared to the legacy Hornet produces a more substantial overall radar signature with greater separation between key scattering centers. When carrying external stores, additional complexity appears in the HRRP, particularly from wing-mounted pylons, fuel tanks, and ordnance. The distinctive nose radome shape and leading edge root extensions create characteristic returns at frontal aspects, making the Super Hornet readily identifiable even when its twin-tail configuration is not prominently visible in the profile.

    \item[\textbf{F22}]
    The F-22 Raptor's HRRP signature exemplifies advanced stealth technology, engineered to minimize strong radar returns through careful edge alignment, surface faceting, and extensive use of radar-absorbent materials. The aircraft's aligned edges are specifically designed to reduce edge diffraction, while its serpentine engine inlets conceal the highly reflective compressor faces that would otherwise create strong returns. Despite these reductions, the HRRP still reveals subtle signatures from the aircraft's trapezoidal wings, canted twin tails, and chined fuselage, though with significantly lower amplitude than conventional fighters. The internal weapons bays eliminate the normally distinctive returns from external stores, creating a cleaner profile particularly at frontal aspects. The careful alignment of surface panel edges, combined with planform alignment techniques, disperses radar energy away from the receiver, resulting in fewer and less intense scattering centers. At non-frontal aspects, the twin vertical stabilizers may become more prominent in the HRRP, providing a potential recognition feature. The F-22's overall signature is characterized by its relative sparsity of dominant scatterers and lower total radar cross-section compared to 4th generation fighters, with aspect angle significantly affecting the observable profile features.

    \item[\textbf{F35}]
    The F-35 Lightning II presents a highly engineered HRRP signature optimized for low observability through its comprehensive stealth design. Unlike conventional fighters, its profile lacks the strong returns typically associated with external weapons, as its internal weapons bays eliminate these distinctive scattering centers. The aircraft's unique divertless supersonic inlet (DSI) with its characteristic 'bump' design obscures the engine face, significantly reducing what would otherwise be a dominant frontal radar return. The F-35's faceted surfaces and aligned edges are specifically designed to control and redirect radar reflections away from the receiver, resulting in a sparse HRRP with fewer prominent peaks compared to 4th generation fighters. The blended wing-body configuration minimizes corner reflectors at junction points, while the single, canted vertical stabilizer presents a smaller signature than twin-tail designs. The Electro-Optical Targeting System (EOTS) embedded under the nose, rather than housed in an external pod, maintains the clean profile. Each variant (A/B/C) displays subtle HRRP differences, with the B's shorter wingspan and lift fan housing creating unique identifying features, while the C's larger wing area and reinforced structure for carrier operations yield a slightly different scattering pattern despite shared stealth characteristics.

    \item[\textbf{IDF}]
    The AIDC F-CK-1 Ching-Kuo (Indigenous Defense Fighter) presents a distinctive HRRP signature characterized by its unique hybrid design elements. The aircraft's side-mounted twin engine inlets create prominent, closely-spaced scattering centers unlike the single-inlet F-16 or the more widely separated intakes of larger twin-engine fighters. Its twin vertical stabilizers generate characteristic dual returns at appropriate aspect angles, while the moderately swept, mid-mounted wings create strong edge diffraction patterns and form complex corner reflectors at their junction with the fuselage. Though incorporating some radar cross-section reduction features in its blended wing-body design, the F-CK-1 lacks comprehensive stealth characteristics, resulting in relatively clear scattering centers from its conventional external weapons carriage, wing-mounted pylons, and the junction points of its airframe components. The aircraft's compact size (smaller than an F-15 but larger than an F-16) yields a somewhat compressed HRRP with closer spacing between major scattering centers. The distinctive forward fuselage chines and streamlined nose cone create a characteristic leading edge return profile, while the afterburner nozzles generate significant returns from rear aspects. This combination of features makes the F-CK-1's HRRP notably different from either American or European fighter designs despite incorporating elements from both design philosophies.

    \item[\textbf{Global Hawk} (全球鹰)]
    The RQ-4 Global Hawk generates a distinctive HRRP signature dominated by its extraordinary wingspan (comparable to a Boeing 737) despite its relatively small radar cross-section for an aircraft of its size. The extremely high aspect ratio wings create extended edge diffraction patterns along their considerable length, forming a key identifying feature in the range profile. Unlike conventional aircraft, the Global Hawk's unique configuration places its single turbofan engine intake atop the fuselage, creating a characteristic scattering center at a position unusual for most aircraft profiles. The V-tail configuration produces a distinctive twin-peaked return pattern significantly different from conventional tail arrangements. The large, bulbous nose housing advanced radar and sensor systems creates a prominent rounded leading edge return, while the clean, streamlined composite fuselage generates relatively modest intermediate returns between the main scattering centers. The absence of external weapons or fuel tanks results in a less complex HRRP compared to combat aircraft. The Global Hawk's significant size combined with its unconventional configuration (high-mounted engine, V-tail, extremely long wings) creates a radar signature pattern that is readily distinguishable from other UAVs or manned reconnaissance platforms, with the extended wing diffraction pattern being particularly characteristic when viewed from broadside aspects.

    \item[\textbf{Predator} (捕食者)]
    The MQ-1 Predator presents a relatively modest HRRP signature owing to its small physical size and extensive use of composite materials, though several distinctive features aid in its identification. The aircraft's inverted V-tail configuration creates a characteristic dual return pattern different from conventional tails, while its single rear-mounted pusher propeller arrangement eliminates the strong frontal returns typically associated with tractor propeller designs. The slender, elongated fuselage produces a relatively weak but extended central return, punctuated by the distinctive bulge of the forward sensor turret housing electro-optical and infrared systems. The high aspect ratio wings generate edge diffraction patterns along their length, though less pronounced than larger platforms like the Global Hawk. On armed variants, the external hardpoints and weapons (typically Hellfire missiles) introduce additional scattering centers along the otherwise clean wing profile. The permanently exposed, non-retractable landing gear struts contribute small but persistent returns at consistent positions in the profile. The Predator's lightweight composite construction reduces specular reflections, but its fundamentally non-stealthy design still presents recognizable scattering centers from its characteristic shape and configuration, with the forward sensor bulge, inverted V-tail, and high-mounted wings forming the most distinctive elements of its HRRP signature.

    \item[\textbf{EA-18G}]
    The EA-18G Growler's HRRP signature builds upon the Super Hornet's baseline profile with several distinctive modifications specific to its electronic warfare mission. Most prominently, the multiple external jamming pods (typically ALQ-99 or Next Generation Jammer pods) create strong, characteristic scattering centers under the wings and fuselage centerline that are not present on standard F/A-18s. The specialized wingtip pods housing the AN/ALQ-218 tactical jamming receiver system replace conventional missile launch rails, creating distinctive endpoint returns in the HRRP. Like its Super Hornet parent, the Growler features strong returns from its twin engine inlets positioned below the leading-edge extensions, forming well-defined scattering centers. The twin, canted vertical stabilizers generate characteristic dual returns, particularly at oblique aspect angles. Complex corner reflections occur at the wing-fuselage junction points, while the relatively blended wing-body design moderates some returns, though without true stealth features. The numerous antenna arrays and radomes specific to the EA-18G's electronic warfare mission add further complexity to the radar signature not seen on standard F/A-18 variants. These specialized electronic warfare components, combined with the aircraft's Super Hornet airframe features, create a distinctive HRRP that differentiates the Growler from both conventional strike fighters and dedicated electronic warfare platforms.

    \item[\textbf{F2}]
    The Mitsubishi F-2's HRRP signature presents a modified version of the F-16's characteristic profile, but with several distinguishing features resulting from its enlarged airframe and structural differences. Most notably, the F-2's 25\% larger size and significantly increased wing area (approximately 40\% greater than the F-16) create more substantial edge diffraction returns with wider separation between scattering centers. The distinctive compound-sweep wing with its unique planform generates a characteristic diffraction pattern different from the F-16's more conventional swept wing. Like the F-16, the single side-mounted engine intake creates a prominent scattering center, but the F-2's larger intake dimensions and slight geometric differences produce a somewhat different return pattern. The single vertical stabilizer generates a strong characteristic return at appropriate aspect angles. The aircraft's conventional external weapons carriage, especially when equipped with the distinctive ASM-2 anti-ship missile, adds complexity to the range profile not seen on standard F-16 configurations. The extensive use of composite materials (approximately 30\% of the structure) slightly reduces specular reflections from some surfaces, but does not provide stealth characteristics. The forward fuselage chines and nose radome shape create a leading edge return profile that, while similar to the F-16, exhibits subtle differences due to the F-2's larger dimensions and modified contours. This combination of familiar F-16-like features executed at a larger scale with unique Japanese modifications makes the F-2's HRRP recognizably different from its American inspiration.

    \item[\textbf{F15}]
    The F-15 Eagle generates a substantial and highly distinctive HRRP signature characterized by multiple strong scattering centers from its large, angular airframe. The twin, widely-spaced engine intakes mounted high on the fuselage sides create prominent, well-separated returns that form a characteristic 'twin peak' pattern in the profile, especially from frontal and oblique aspects. These large rectangular intakes with their sharp edges and internal compressor faces produce some of the strongest radar returns in the entire profile. The aircraft's twin vertical stabilizers, distinctively canted outward, generate additional dual returns at the aft section of the profile, creating a signature notably different from single-tail fighters. The F-15's broad wing area with its moderate sweep angle produces extended edge diffraction patterns, while the distinct separation between the wing and fuselage creates pronounced corner reflectors at their junction points. The sharp discontinuity between the flat-sided fuselage and the vertical stabilizers forms additional strong corner reflectors. Unlike more modern designs, the F-15 lacks significant radar cross-section reduction features, resulting in a complex HRRP with numerous well-defined scattering centers. When carrying external fuel tanks and weapons, additional distinct returns appear along the profile. The F-15's substantial size, angular design philosophy with limited use of curved surfaces, and distinctive twin-intake/twin-tail configuration make its HRRP signature one of the most recognizable among modern fighter aircraft.

    \item[\textbf{F16}]
    The F-16 Fighting Falcon produces a distinctive HRRP signature dominated by its single side-mounted engine intake, which creates a prominent asymmetric scattering center unlike the twin-intake designs of many contemporaries. This large intake with its sharp-edged boundary and internal engine face generates one of the strongest individual returns in the profile, especially from frontal and oblique aspects. The F-16's blended wing-body design reduces some corner reflector effects compared to more angular fighters, but the junction points still create identifiable returns. The moderately swept wings with their characteristic leading edge extensions generate edge diffraction patterns of moderate intensity. The single vertical stabilizer produces a distinctive central return at the aft section of the profile, differentiating it from twin-tail designs like the F-15 or F/A-18. The aircraft's compact size results in a relatively condensed HRRP with closer spacing between major scattering centers compared to larger fighters. The distinctive spherical radome and smooth nose contours create a characteristic leading edge return profile. When carrying external stores, additional complexity appears in the range profile from wing-mounted pylons, fuel tanks, and weapons. The ventral fins found on some variants add small but consistent returns to the lower fuselage region. This combination of a dominant single-intake return, moderate-sized airframe, and single-tail configuration makes the F-16's HRRP readily distinguishable from larger or twin-engine fighters despite its relatively modest overall radar cross-section.

    \item[\textbf{Migrate 2000} (幻影2000)]
    The Dassault Mirage 2000's HRRP signature is distinctively characterized by its pure delta wing configuration, which produces a unique triangular edge diffraction pattern unlike the more conventional swept-wing fighters. The absence of horizontal stabilizers creates a cleaner aft profile with fewer scattering centers compared to conventional tail configurations. The aircraft's semi-circular air intake positioned under the fuselage generates a prominent central scattering center, distinctly different from the side-mounted intakes of the F-16 or the twin widely-spaced intakes of the F-15. The single vertical stabilizer produces a characteristic central return at the aft section. The Mirage 2000's forward fuselage strakes create small but consistent edge diffraction returns that add to the profile's distinctiveness. The delta wing's sharp leading edges produce strong diffraction effects, particularly at broadside aspects, while the wing-fuselage junction forms complex corner reflectors despite some blending. The relatively compact airframe results in a somewhat compressed HRRP with closer spacing between major scattering centers compared to larger fighters. The distinctive nose chine and radome shape create a characteristic leading edge return profile. When carrying external weapons and fuel tanks, additional distinct returns appear along the profile, though the clean delta configuration still remains recognizable. This combination of pure delta wing configuration without separate horizontal stabilizers, under-fuselage intake, and relatively clean design philosophy gives the Mirage 2000 an HRRP signature readily distinguishable from American or Russian fighter designs.

    \item[\textbf{An-26}]
    The Antonov An-26 transport aircraft generates a substantial HRRP signature characterized by its large, boxy fuselage with minimal radar cross-section reduction features. The aircraft's straight, high-mounted wings create strong edge diffraction patterns and form pronounced corner reflectors at their junction with the rectangular fuselage. The twin turboprop engines with their prominent nacelles extending well forward of the wing leading edge create distinctive and strong scattering centers on either side of the fuselage. The flat sides and bottom of the cargo compartment produce intense specular reflections when perpendicular to the radar line-of-sight. The conventional tail with its single vertical stabilizer and horizontal stabilizers generates additional strong returns at the aft section of the profile. The distinctive glazed navigator station in the nose, with its numerous flat panels and angular surfaces, creates a complex pattern of returns at the forward section unlike the smoother radomes of combat aircraft. The rear loading ramp with its potential for cavity returns adds further complexity to the aft profile. The exposed landing gear fairings, even when the gear is retracted, contribute additional small but consistent returns. Various antennas and external protrusions add minor scattering centers across the airframe. The An-26's large size, predominantly rectangular cross-section, minimal use of composite materials or radar-absorbing structures, and numerous sharp edge discontinuities combine to create a complex HRRP with multiple strong scattering centers characteristic of Soviet-era military transport design philosophy.

    \item[\textbf{Yark-42}] % Changed key to match mapping
    The Yakovlev Yak-42 presents a distinctive HRRP signature stemming from its unusual three-engine configuration with all power plants mounted at the rear fuselage. Two engines are mounted on the sides of the rear fuselage while the third is integrated into the base of the vertical stabilizer, creating a unique cluster of three closely-spaced scattering centers at the aft section unlike more common twin-engine designs. The aircraft's T-tail configuration produces a characteristic elevated horizontal return above the main fuselage line. The low-mounted swept wings generate edge diffraction patterns and form corner reflectors at their junction with the fuselage. The clean, cylindrical fuselage creates a relatively uniform central return until interrupted by the wing junction and rear-mounted engines. The distinctive nose profile with its slightly downward slope differs from Western airliners, creating a recognizable leading edge return. The conventional airliner design philosophy, with no stealth features, results in numerous clear scattering centers from various airframe components and discontinuities. The passenger windows create a series of small, regularly-spaced returns along the fuselage sides, though these may blend into the overall fuselage return at lower range resolutions. The wing-mounted landing gear bays, when closed, create subtle discontinuities in the otherwise clean wing profile. This combination of T-tail configuration, tri-jet engine arrangement clustered at the rear, and Soviet-era airliner design characteristics makes the Yak-42's HRRP readily distinguishable from Western twin-engine commercial aircraft despite similar overall sizes.

    \item[\textbf{Cessna Citation}]
    The Cessna Cessna Citation family of business jets presents HRRP signatures characterized by their compact size and clean, conventional design. The defining feature across most Cessna Citation models is the twin rear-mounted turbofan engines flanking the aft fuselage, creating distinctive paired scattering centers at the rear of the profile. The T-tail configuration produces a characteristic elevated horizontal return above the main fuselage line, similar to other T-tail aircraft but on a smaller scale than commercial airliners. The low-mounted, moderately swept wings generate modest edge diffraction patterns and form corner reflectors at their junction with the sleek fuselage. The relatively small fuselage diameter creates a more modest central return compared to larger commercial aircraft. The smooth, gradually tapered nose section houses the weather radar and produces a clean leading edge return without the complex cockpit windscreen reflections of older designs. The absence of external stores or numerous antennas results in a cleaner overall profile than military aircraft. Various Cessna Citation models differ in length and wingspan, affecting the spacing between scattering centers, but maintain the same basic configuration of rear-mounted engines and T-tail. The relatively small size compared to commercial airliners results in a compressed HRRP with closer spacing between major features. Landing gear bays create minor discontinuities when closed, while extended gear adds complexity to the lower profile. This combination of compact business jet proportions with conventional T-tail and twin rear-mounted engine configuration creates an HRRP signature that is characteristic of the Cessna Citation family while being distinguishable from larger commercial aircraft or front-engine business jets.

\end{description}}

% --- DeepSeek-R1 Descriptions ---
\section{DeepSeek-R1 生成的描述}
\label{sec:deepseek_desc}
\footnotesize
{\begin{description}
    \item[\textbf{EP-3E}]
    The EP-3E Aries II, a signals intelligence variant of the P-3 Orion, presents a complex HRRP signature dominated by four wing-mounted Allison T56 turboprop engines. The large-diameter propellers and nacelles generate periodic modulation and strong cavity scattering from their inlets, while the straight high-aspect-ratio wings produce edge diffraction returns. A prominent bulbous nose radome housing the AN/APS-134 radar creates a forward scattering center, contrasting with the angular returns from the dorsal 'football' SIGINT antenna and ventral blade arrays. The conventional tail assembly, with horizontal stabilizers mounted mid-fin, forms corner reflectors distinct from T-tailed transports. The elongated fuselage and multiple protruding sensors yield a distributed profile with secondary peaks from structural joints and blade antennas.

    \item[\textbf{F18}]
    The F/A-18F Super Hornet’s HRRP features twin dominant returns from its trapezoidal engine inlets positioned beneath pronounced leading-edge extensions (LEX), creating corner reflector effects at the wing-body junction. Unlike legacy Hornets, its enlarged LEX and trapezoidal intakes increase cavity scattering, while canted twin vertical stabilizers produce asymmetric dual peaks at oblique aspects. The sawtooth trailing edges on wings and stabilizers reduce traveling wave scattering but enhance edge diffraction. External conformal fuel tanks and wingtip missile rails add secondary scatterers absent in clean configurations, differentiating it from single-engine fighters like the F-16.

    \item[\textbf{F22}]
    The F-22’s stealth-optimized geometry suppresses HRRP amplitude through diamond-shaped wing planforms, aligned edges, and S-curved engine inlets masking fan blades. Primary scatterers include the canopy’s forward-facing dihedral and weapon bay edges, while canted vertical tails create aspect-dependent corner reflections. Unlike conventional fighters, its internal weapons bays eliminate external store signatures, and radar-absorbent material (RAM) coating attenuates cavity resonances from serpentine inlets. The profile shows fewer dominant peaks but retains identifiable spacing between wing-root chines and tail surfaces at specific aspect angles.

    \item[\textbf{F35}]
    The F-35’s diverterless supersonic inlets (DSI) eliminate traditional inlet boundary layer ducts, reducing cavity scattering but creating characteristic forward bumps visible in nose-on HRRP. Chined forebody edges and faceted canopy contribute to low-observable specular returns, while the single vertical tail’s swept leading edge produces a traveling wave peak. Unlike the F-22, its engine exhaust lacks thrust vectoring vanes, simplifying aft-sector profiles. External pylons (when mounted) introduce abrupt scattering contrasts compared to clean configurations, with the F-35B’s lift fan door creating a unique mid-fuselage discontinuity.

    \item[\textbf{IDF}]
    Taiwan’s F-CK-1 combines F-16-like LEX with twin Garret TFE1042 engines in bifurcated intakes, producing dual cavity resonances absent in single-engine fighters. The swept vertical stabilizer’s root forms a corner reflector with the dorsal spine, while ventral strakes beneath the tail create additional edge diffraction. Wingtip launchers for Sky Sword missiles act as persistent scatterers, contrasting with F-16s’ cleaner wingtips. The absence of conformal fuel tanks (compared to later F-16 variants) leaves a smoother fuselage return profile between wing and tail sections.

    \item[\textbf{Global Hawk} (全球鹰)]
    This high-altitude UAV’s HRRP is characterized by a slender forward fuselage with a protruding sensor turret creating a forward scattering peak, followed by a low-RCS chined dorsal spine. The inverted V-tail minimizes corner reflections compared to conventional tails, while the rear-mounted Rolls-Royce AE3007 turbofan’s intake and exhaust produce merged returns. The high-aspect-ratio wings exhibit edge diffraction along their full span, with wingtip pods (housing satellite antennas) adding distinct end-region scatterers absent in smaller UAVs like the MQ-1.

    \item[\textbf{Predator} (捕食者)]
    The MQ-1’s pusher-configuration Rotax piston engine and inverted V-tail create a compact aft-sector HRRP dominated by the propeller’s rotational modulation and tail boom joint discontinuities. The forward sensor turret under the bulbous nose generates a spherical scattering center, while straight wings with dihedral produce consistent edge returns. Unlike jet-powered UAVs, the absence of engine inlets simplifies the mid-profile, but wing-mounted pitot probes and communications antennas introduce minor secondary peaks.

    \item[\textbf{EA-18G}]
    Derived from the F/A-18F, the EA-18G’s HRRP is distinguished by ALQ-218 wingtip jamming pods creating persistent high-amplitude returns, contrasting with clean Hornet wingtips. The addition of ventral AN/ALQ-99 pods introduces underbelly scatterers absent in other Super Hornet variants. Retained LEX and intake features produce similar forward-sector peaks to the F/A-18F, but the dorsal spine’s electronic warfare antennas add mid-profile scattering not seen in strike fighters.

    \item[\textbf{F2}]
    Japan’s F-2, based on the F-16, exhibits a larger wing area with composite leading edges creating enhanced edge diffraction compared to the F-16’s metal wings. The chin-mounted intake’s splitter plate generates a strong forward scatterer, while conformal fuel tanks along the fuselage sides add mid-profile returns absent in early F-16s. Wingtip AAM-3 missile rails act as fixed scatterers, differing from the F-16’s launcher-adaptive wingtips. The J/APG-1 radar’s larger antenna increases nose-sector RCS relative to F-16 variants.

    \item[\textbf{F15}]
    The F-15’s HRRP features widely spaced engine intakes creating two dominant forward-sector returns, with flat side panels between intakes and wings forming large specular surfaces. Twin vertical tails angled 15° outward produce aspect-dependent dual peaks contrasting with single-tail fighters. The lack of leading-edge extensions (compared to F/A-18) simplifies the wing-root profile, while external fuel tank pylons (when mounted) introduce periodic scattering centers along the wings.

    \item[\textbf{F16}]
    The F-16’s chin-mounted intake creates a single prominent forward scattering center, differing from twin-engine fighters’ bifurcated returns. Leading-edge extensions (LEX) form a corner reflector with the wing roots, while the bubble canopy generates a curved-surface return absent in faceted stealth aircraft. The single vertical stabilizer’s swept leading edge produces traveling wave scattering, with ventral strakes adding secondary aft-sector peaks. Clean configurations lack wingtip scatterers present in later variants with integrated launchers.

    \item[\textbf{Migrate 2000} (幻影2000)]
    The delta-wing Mirage 2000’s HRRP shows continuous leading-edge diffraction along its 58° swept wings, creating extended returns absent in conventional wing fighters. The semi-circular engine intake’s centerbody radar cone acts as a forward scatterer, while the single vertical tail’s junction with the dorsal spine forms a corner reflector. Unlike canard-equipped fighters, its tailless design eliminates horizontal surface interactions, but wingtip ECM pods add distinct end-region returns during operational deployments.

    \item[\textbf{An-26}]
    This turboprop transport’s HRRP is dominated by wing-mounted Ivchenko AI-24 engines, with large propellers causing blade flash modulation and nacelles creating cavity scattering. The high-wing configuration positions engine returns above the fuselage’s boxy structure, which generates corner reflections at cargo door edges. Twin vertical stabilizers create closely spaced aft peaks, while the glassed-in navigator’s nose station produces a spherical return distinct from radar-equipped military transports.

    \item[\textbf{Yark-42}]
    The Yak-42 trijet’s rear-mounted Lotarev D-36 engines cluster three intake/exhaust signatures in the aft sector, creating merged high-amplitude returns. The T-tail configuration forms a prominent corner reflector at the vertical/horizontal stabilizer junction, while the low-wing design positions engine nacelle scatterers below the fuselage. The elongated nose radome and forward entry door edges introduce secondary forward peaks absent in shorter regional jets. Swept wings with winglets produce edge diffraction peaks along their entire span.

    \item[\textbf{Cessna Citation}]
    T-tailed Cessna Citations like the CJ series exhibit a rear-engine configuration where Honeywell TFE731 turbofans mounted on the fuselage sides create dual aft-sector scatterers. The T-tail itself forms a right-angle corner reflector distinct from conventional tail business jets. Straight wings with winglets generate edge diffraction peaks at both leading and trailing edges, while the oval fuselage cross-section minimizes flat-plate specular returns compared to corporate jets with angular fuselages.

\end{description}}

% --- GPT-4.1 Descriptions ---
\section{GPT-4.1 生成的描述}
\label{sec:gpt41_desc}
\footnotesize
{\begin{description}
    \item[\textbf{EP-3E}]
    The Lockheed EP-3E Aries II presents an HRRP distinguished by its elongated fuselage and high-mounted wings, with four prominent wing-mounted turboprop engines generating strong, regularly spaced scattering centers. The extended nose, housing specialized signals intelligence equipment, is a key forward scatterer, while the tail-mounted MAD boom creates a distinctive rearward reflection. Numerous antenna arrays and fairings distributed along the fuselage introduce unique, irregular diffraction patterns and localized high-intensity radar returns, making the EP-3E's signature easily distinguishable from standard maritime patrol aircraft. These modifications contribute to a more complex HRRP with multiple closely spaced peaks, emphasizing the structural and mission-specific features that set it apart from conventional P-3 Orion variants.

    \item[\textbf{F18}]
    The Boeing F/A-18 Hornet's HRRP signature is defined by its mid-sized twin-engine configuration, with paired engine inlets beneath leading-edge extensions (LEX) producing unique, strong corner reflections. The aircraft's twin canted vertical stabilizers yield parallel, spaced scattering centers, a discriminating feature compared to single-tail fighters. The shoulder-mounted intakes and moderate wing sweep contribute prominent edge diffractions and multipath returns, while the blended wing geometry and LEX structures further enhance edge scattering effects, resulting in a distinctive, multi-peaked HRRP profile that is readily separable from other tactical fighters.

    \item[\textbf{F22}]
    The Lockheed Martin F-22 Raptor's HRRP is dominated by its stealth-optimized geometry, resulting in minimal, carefully controlled scattering centers. Advanced edge alignment, chined fuselage, and internal weapons bays reduce specular reflections, while the planform alignment of leading and trailing edges redirects energy away from the radar. The S-duct inlets and use of radar absorbent materials (RAM) suppress engine and intake returns, confining dominant peaks to a few managed locations. The overall HRRP demonstrates low amplitude and a lack of periodic, strong returns typical of conventional fighters, with characteristic angularly-sensitive diffuse scattering that distinguishes the F-22 even in sparse radar data.

    \item[\textbf{F35}]
    The Lockheed Martin F-35 Lightning II's HRRP is notable for its single, embedded engine configuration, serpentine intake ducts, and stealth-focused edge alignment, all of which minimize dominant scattering centers. Internal weapon bays, faceted surfaces, and radar absorbent materials result in a diffuse, low-intensity return pattern. The overall signature consists of subtly distributed peaks, with faint reflections from the nose radome and intake lips, and, in the case of the F-35B, potential minor returns from the lift fan housing. Compared to non-stealth aircraft, the F-35's HRRP is marked by reduced amplitude, fewer strong peaks, and a smoother, less cluttered profile that maintains stealth characteristics across various radar bands.

    \item[\textbf{IDF}]
    The AIDC F-CK-1 Ching-Kuo (IDF) generates an HRRP characterized by its compact airframe, twin closely-spaced vertical stabilizers, and side-mounted engine inlets, all of which create multiple distinct but compact scattering centers. The moderately swept wings, with blended root extensions, produce pronounced edge diffractions, while the hybrid design—incorporating elements from both F-16 and F-18—results in unique corner reflections and multiple peaks not present in its American counterparts. The twin-engine configuration further distinguishes its HRRP by producing paired returns from the exhausts and intakes, setting the IDF apart in radar signature databases.

    \item[\textbf{Global Hawk} (全球鹰)]
    The RQ-4 Global Hawk's HRRP is dominated by its exceptionally long, high-aspect-ratio wings, which generate strong, widely spaced leading-edge reflections. The bulbous nose, housing sophisticated sensor suites, produces a prominent forward scattering center, while the dorsal-mounted single turbofan engine creates a mid-fuselage reflection distinct from conventional rear- or wing-mounted engine placements. The V-tail configuration introduces a unique set of corner reflections at the aft end. The smooth composite fuselage and absence of external stores result in a relatively uncluttered HRRP, where the main discriminants are the spatial separation of wing, engine, and V-tail returns, differentiating the Global Hawk from both manned aircraft and smaller UAVs.

    \item[\textbf{Predator} (捕食者)]
    The MQ-1 Predator's HRRP reflects its slender fuselage and high-aspect-ratio, straight wing design, leading to a narrow but elongated pattern of wing-edge diffractions. The bulbous nose, which houses the primary sensor ball, forms a distinct forward scattering center. Its pusher propeller configuration at the tail provides intermittent flash returns when the blades align with the radar pulse, and the inverted V-tail introduces unique corner reflections at the aft end. The absence of external stores generally yields a sparse HRRP, with its key scattering points—nose, wing roots/tips, and tail—being well-separated and highly characteristic of this UAV.

    \item[\textbf{EA-18G}]
    The Boeing EA-18G Growler's HRRP is uniquely characterized by strong returns from its wingtip-mounted ALQ-99 jamming pods, which create additional edge diffraction and localized scattering centers not present in standard F/A-18 variants. Retaining the Super Hornet's twin-engine and twin vertical stabilizer configuration, the Growler's signature also features unique antenna arrays mounted on the fuselage and tails. Centerline electronic warfare pods contribute further concentrated returns, resulting in a more complex, multi-peaked HRRP with lateral symmetry and additional peaks, making the EA-18G readily distinguishable from baseline F/A-18s and other tactical jets.

    \item[\textbf{F2}]
    The Mitsubishi F-2's HRRP reflects its F-16-derived airframe but is marked by its 25\% larger wing area, which creates broader, more widely separated wing-edge diffraction returns. The enlarged nose radome produces a strong, forward scattering center, while the expanded, compound-curved canopy introduces additional specular reflections. The longer fuselage and enlarged vertical stabilizer further increase the radar cross-section, resulting in a profile with higher amplitude frontal and side returns. The co-cured composite wing structure contributes unique diffraction patterns, making the F-2's HRRP both broader and more complex than its American progenitor.

    \item[\textbf{F15}]
    The McDonnell Douglas F-15 Eagle's HRRP is dominated by its large, widely spaced twin-engine nacelles and prominent rectangular air intakes, producing strong, periodic corner reflections at the mid-fuselage. Its broad wingspan, squared wingtips, and twin vertical stabilizers generate multiple, high-intensity returns, with the vertical tails yielding parallel linear peaks toward the rear of the profile. Numerous surface discontinuities, right-angle joins, and exposed weapons hardpoints contribute to a cluttered HRRP, with multiple strong scattering centers that make the F-15's signature unmistakable compared to more compact or stealth-optimized fighters.

    \item[\textbf{F16}]
    The General Dynamics F-16 Fighting Falcon's HRRP signature is defined by its compact, blended wing-body design and single ventral air intake, which forms a dominant, centrally located scattering center. The mid-mounted, moderately swept wings create strong leading-edge diffractions, while the single vertical stabilizer produces an asymmetric rearward return, especially at oblique angles. The combination of a small radar cross-section, bubble canopy, and the absence of twin-engine or twin-tail features yields a relatively short, focused HRRP profile, punctuated by recognizable reflection patterns that enable discrimination from larger or multi-engine fighters.

    \item[\textbf{Migrate 2000} (幻影2000)]
    The Dassault Mirage 2000's HRRP is characterized by its pure delta wing, resulting in a distinctive triangular pattern of strong leading-edge diffractions. The absence of horizontal stabilizers creates a clean rear profile, while the single vertical tail produces an asymmetric, isolated reflection at the aft end. The semi-recessed air intakes on either side of the fuselage generate unique corner reflections, and in canard-equipped variants, forward scattering centers add additional complexity. Compared to other delta-wing designs, the Mirage 2000's HRRP stands out for its geometric simplicity, pronounced wing returns, and distinct lack of tailplane-associated peaks.

    \item[\textbf{An-26}]
    The Antonov An-26's HRRP is defined by its high-wing, twin-engine turboprop configuration, with strong, widely spaced scattering centers from the large engine nacelles beneath the wings. The upward-sloping rear fuselage and rear loading ramp generate a pronounced, extended reflection toward the aft section, while the boxy fuselage and T-tail configuration introduce clear, separated peaks at the tail. The clustering of dominant returns—nose, engines, wing roots, and ramp—results in a stretched and highly structured HRRP, making the An-26 distinguishable from jet-powered or low-wing transports.

    \item[\textbf{Yark-42}]
    The Yakovlev Yak-42's HRRP is shaped by its medium-range trijet layout, with three closely clustered rear-mounted engines providing concentrated, strong scattering centers at the aft end of the profile. The T-tail configuration introduces a prominent vertical return, while the moderately swept low-mounted wings contribute additional, but less pronounced, edge diffractions. The cylindrical fuselage yields a smooth, persistent mid-profile return, and the unique tail-mounted engine arrangement distinguishes the Yak-42's HRRP from twin-jet or conventional airliner designs.

    \item[\textbf{Cessna Citation}]
    The Cessna Cessna Citation business jet family's HRRP typically features a compact, low-wing design with a T-tail, and two rear-fuselage-mounted turbofan engines forming distinct, symmetrical aft scattering centers. The small overall size leads to closely spaced peaks in the HRRP, with the main returns arising from the nose radome, engine nacelles, and tailplane junctions. The clean, uninterrupted fuselage and lack of external stores result in a simple, uncluttered signature that is easily distinguishable from larger commercial airliners or unconventional business jet layouts.

\end{description}}

% --- Gemini-2.5-pro Descriptions ---
\section{Gemini-2.5-pro 生成的描述}
\label{sec:gemini_desc}
\footnotesize
{\begin{description}
    \item[\textbf{EP-3E}]
    The EP-3E Aries II, derived from the P-3 Orion airframe, presents an HRRP signature dominated by its four large turboprop engines mounted on its low, straight wing. Strong scattering contributions arise from the engine nacelles, propellers (creating complex, potentially time-varying returns), and exhausts. Significant corner reflections are expected between the wing undersides and the long, cylindrical fuselage, as well as between the conventional tail surfaces (vertical stabilizer and horizontal stabilizers) and the fuselage. Numerous distinctive external fairings, antennas, and the large ventral 'canoe' radome associated with its SIGINT mission add unique scattering centers along the fuselage length, increasing the profile's complexity and providing key discriminative features compared to standard P-3s or other four-engine turboprops.

    \item[\textbf{F18}]
    The F/A-18's (specifically Super Hornet, F/A-18E/F, being more distinct) HRRP is characterized by prominent returns from its large, angular, side-mounted engine inlets located beneath the leading-edge extensions (LEX). These inlets, along with the engine compressor faces (if visible to the radar) and exhausts, form dominant scattering centers. The twin, canted vertical stabilizers generate distinct dual returns, particularly at non-cardinal aspect angles, differentiating it from single-tail fighters. Strong corner reflections occur at the wing-fuselage junctions and tail-fuselage junctions. Edge diffraction from the moderately swept wings, LEX, and horizontal stabilizers adds further structure to the profile. Compared to the legacy Hornet (F/A-18C/D), the Super Hornet's larger size, more angular inlets, and larger wing/LEX area modify the relative spacing and intensity of these scattering features.

    \item[\textbf{F22}]
    As a premier stealth fighter, the F-22 Raptor's HRRP is intentionally designed for low observability, resulting in significantly reduced amplitude and fewer dominant peaks compared to conventional aircraft. Key contributors to its low signature include serpentine (S-duct) engine inlets that mask the highly reflective engine compressor faces, extensive use of radar-absorbent materials (RAM), and internal weapons carriage eliminating external store scattering. The design heavily utilizes edge alignment, where wing and tail leading/trailing edges are parallel, potentially causing synchronized specular flashes or edge diffraction returns only at very specific, narrow aspect angles. Faceting on the fuselage and the canted twin vertical tails further deflect radar energy away from the source. Residual scattering may originate from leading edges, tail edges, the faceted canopy/nose section, and potentially complex returns from the thrust-vectoring nozzles, but these are minimized and highly aspect-dependent.

    \item[\textbf{F35}]
    Similar to the F-22, the F-35 Lightning II is a stealth aircraft exhibiting a low-observable HRRP signature characterized by reduced amplitude and suppressed peaks. Its single engine features a diverterless supersonic inlet (DSI) and serpentine duct, significantly masking engine face returns. Internal weapons bays eliminate external scatterers. Extensive use of RAM, edge alignment (though potentially less rigorous than F-22), and blended shaping contribute to its low RCS. Key scattering points, though minimized, likely arise from edge diffraction off the wing and tail surfaces (including the unique horizontal/vertical tail combination), specular returns from specifically angled facets, the DSI bump, and the electro-optical targeting system (EOTS) aperture under the nose. Its signature is highly aspect-dependent, designed to concentrate reflections away from the radar source across most angles.

    \item[\textbf{IDF}]
    The AIDC F-CK-1 Ching-Kuo presents an HRRP characteristic of a compact, twin-engine light fighter. Significant returns are expected from the two closely spaced afterburning turbofan engine inlets located under the forward fuselage/wing roots and their corresponding exhausts. The mid-mounted wings with leading-edge root extensions (LERX) create notable edge diffraction and corner reflection points with the fuselage. Its conventional single vertical stabilizer and horizontal stabilizers contribute distinct scattering centers through edge diffraction and corner effects at the tail-fuselage junction. The relatively simple fuselage shape and pointed radome also contribute to the overall profile. Its signature complexity lies between simpler single-engine fighters and larger twin-engine designs.

    \item[\textbf{Global Hawk} (全球鹰)]
    The RQ-4 Global Hawk's HRRP is unique due to its very large wingspan (comparable to a Boeing 737) and slender fuselage housing a single turbofan engine mounted dorsally near the tail. The engine inlet and exhaust, despite being relatively small, are key scatterers due to their placement. The most dominant features are likely the extensive edge diffraction returns from the extremely high-aspect-ratio wings. Strong corner reflections are expected between the wing roots and the fuselage, and between the prominent V-tail surfaces and the fuselage. The large, bulbous forward fuselage housing the primary sensor payload and radome constitutes a significant scattering region at the front of the profile. Secondary scatterers may include various ventral sensor pods and antennas depending on the specific variant/payload.

    \item[\textbf{Predator} (捕食者)]
    The MQ-1 Predator UAV's HRRP is characterized by its relatively small size and unique configuration. A dominant scattering center arises from the rear-mounted piston engine and its pusher propeller, creating a complex signature at the profile's end, distinct from jet aircraft. The slender fuselage and long, straight, high-aspect-ratio wings produce edge diffraction returns. Corner reflections between the wing roots and fuselage, and particularly at the inverted V-tail junction with the fuselage, are significant contributors. The prominent sensor turret typically mounted under the nose provides a distinct scattering peak at the front of the HRRP signature. Its overall lower RCS and unique pusher-prop/inverted V-tail configuration make it distinguishable from larger UAVs and manned aircraft.

    \item[\textbf{EA-18G}]
    The EA-18G Growler shares the basic airframe and thus many HRRP characteristics with the F/A-18 Super Hornet, including strong returns from the angular inlets, twin canted tails, wing/fuselage corner reflections, and edge diffraction from wings/LEX. However, its unique electronic attack mission equipment provides critical distinguishing features in the HRRP. The most prominent additions are the ALQ-99 tactical jamming pods typically carried on the wingtips and potentially underwing pylons. These large pods act as significant additional scattering centers, effectively extending the apparent span-wise dimension in the HRRP and adding distinct peaks not present on standard F/A-18s. Other ESM/ECM antennas and fairings may add further subtle complexity to the profile.

    \item[\textbf{F2}]
    The Mitsubishi F-2, closely resembling a scaled-up F-16, exhibits an HRRP influenced by its single, large ventral engine inlet (a key scattering feature) and the corresponding engine exhaust. Its larger wing area compared to the F-16, with moderate sweep, generates significant edge diffraction returns and strong corner reflections where the wings blend into the fuselage. The conventional single vertical tail and horizontal stabilizers contribute further edge diffraction and corner scattering at the aft section. The blended wing-body design and potentially different material choices or minor shaping details compared to the F-16 might introduce subtle differences in scattering intensity and distribution along the profile. The canopy and nose radome also contribute to the forward part of the signature.

    \item[\textbf{F15}]
    The F-15 Eagle's HRRP is dominated by the prominent returns from its two large, side-mounted rectangular engine inlets and the twin engine exhausts. These features create strong, distinct scattering centers. The high-mounted wings generate significant edge diffraction and corner reflections with the fuselage sides. A defining characteristic is its twin vertical stabilizers, which, along with the horizontal stabilizers (stabilators), create complex corner reflection and diffraction effects at the rear, differing significantly from single-tail designs like the F-16. The relatively large fuselage and distinct bubble canopy also contribute notable scattering to the overall range profile.

    \item[\textbf{F16}]
    The F-16 Fighting Falcon's HRRP signature is strongly influenced by its single, distinctive ventral engine inlet (chin inlet), which acts as a primary scattering center at the front-underside aspect. The single engine exhaust nozzle is another major contributor at the rear. Its moderately swept, blended wing-body design produces significant edge diffraction from the wing's leading and trailing edges and corner reflections at the wing-fuselage junction. The single, tall vertical stabilizer and horizontal stabilizers (stabilators) create characteristic corner and edge scattering effects at the tail, distinguishing it from twin-tail designs. The prominent bubble canopy is also a notable scatterer.

    \item[\textbf{Migrate 2000} (幻影2000)]
    The Dassault Mirage 2000 is characterized by its large, tailless delta wing configuration. This results in a unique HRRP where dominant returns likely originate from the semi-circular engine air intakes located on the fuselage sides near the wing roots, and the single engine exhaust. The vast expanse of the delta wing's leading edge generates very strong edge diffraction returns, particularly at near head-on aspects. The lack of horizontal stabilizers eliminates tail-fuselage corner reflections found in conventional designs, but the large single vertical stabilizer creates its own corner reflection with the fuselage and contributes edge diffraction. The sharply swept leading edge and overall shape produce a signature distinct from non-delta wing fighters.

    \item[\textbf{An-26}]
    The Antonov An-26, a twin-engine turboprop transport, exhibits an HRRP dominated by scattering from its two large engine nacelles mounted on its high wings, including returns from the propellers and exhausts. The high-wing configuration creates a very strong corner reflector effect between the wing undersides and the cylindrical fuselage. Significant edge diffraction occurs along the leading and trailing edges of the relatively straight wings and the conventional tail surfaces (single vertical stabilizer, horizontal stabilizers). The fuselage itself, particularly the nose radome and the distinctive rear cargo ramp assembly (even when closed), contributes additional scattering centers along the profile length, resulting in a relatively complex, non-stealthy signature.

    \item[\textbf{Yark-42}]
    The Yakovlev Yak-42 regional jet presents a distinctive HRRP due to its tri-jet configuration with two engines pylon-mounted on the rear fuselage sides and a third engine housed internally at the base of the T-tail, fed by an S-duct inlet. Strong scattering centers arise from the inlets and exhausts of all three engines, with the two side-mounted engines being particularly prominent. The T-tail configuration itself is a major feature, creating strong corner reflections between the vertical stabilizer and the high-mounted horizontal stabilizer, as well as between the vertical stabilizer base/S-duct inlet and the fuselage. The moderately swept wings contribute edge diffraction and wing-fuselage corner reflections. This engine and tail arrangement creates a unique distribution of major scattering centers compared to twin-engine or four-engine designs.

    \item[\textbf{Cessna Citation}]
    The Cessna Cessna Citation family of business jets typically shares common HRRP characteristics stemming from their configuration: twin turbofan engines mounted high on the aft fuselage or tail cone, and often a T-tail arrangement (though some variants have conventional tails). Dominant scattering centers are the engine inlets and exhausts, positioned closely together at the rear. The T-tail (on applicable models) creates a significant corner reflector effect between the vertical and horizontal stabilizers, high above the fuselage return. The relatively small, often low-mounted, straight or slightly swept wings contribute edge diffraction and wing-fuselage corner reflections. The rounded nose/radome and smoothly contoured fuselage provide additional, less complex scattering contributions. While varying in size, this rear-engine/T-tail combination provides a characteristic signature pattern for many members of the Cessna Citation family.

\end{description}}

%----------------------------------------------------------------------
% End of Appendix Chapter
%----------------------------------------------------------------------
