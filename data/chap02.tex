\chapter{小样本HRRPRATR 技术基本原理}
\label{chap:theory}

\section{引言}
\label{sec:theory_intro}

本章的核心目标是为后续章节所提出的基于元学习的小样本HRRP识别新方法奠定必要的理论基础和数学铺垫。具体而言,本章将首先从电磁散射理论和雷达信号处理的基本原理出发,建立宽带雷达信号模型以及HRRP成像的数学推导过程,并结合典型空天目标的电磁散射特性分析,阐明影响HRRP形态和性质的关键物理因素。在此过程中,我们将引入噪声和杂波对HRRP影响的数学模型,并进一步形式化描述HRRP的角度敏感性,为后续章节应对这些问题提供理论依据。其次,我们将回顾并形式化地描述基于深度学习的RATR技术框架,介绍几种在HRRP识别中常用的典型深度学习模型(如卷积神经网络CNN、循环神经网络RNN/长短时记忆网络LSTM)的数学结构及其适用性。再次,也是本章的重点内容,我们将对FSL问题进行严格的数学定义,并重点介绍作为解决FSL问题主流范式的元学习的基本框架,特别是基于度量学习(Metric-based Meta-Learning)和基于优化(Optimization-based Meta-Learning)两种主要技术路线的数学原理和算法流程。在讨论FSL时,我们将结合HRRP的特性,阐述小样本条件下特征判别性不足问题的根源,并从模型输入角度形式化语义信息利用匮乏的问题。最后,对本章内容进行归纳总结。本章所建立的数学模型、形式化定义、基础理论和相关术语将作为后续各章节研究工作的基础,为新算法的理解、设计、实现和性能评估提供支撑。

\section{空天目标宽带雷达成像机理}
\label{sec:hrrp_mechanism}

HRRP的形成是雷达系统发射宽带信号与目标发生电磁相互作用并经接收处理的结果。其精细结构蕴含了目标沿雷达视线(Line of Sight, LOS)方向的散射中心分布信息,是目标识别的重要依据。

\subsection{宽带雷达信号模型与HRRP成像}
\label{subsec:hrrp_imaging_model}

为了获得高的距离分辨率 $\Delta R$,现代雷达通常发射具有大带宽 $B$ 的信号。除了线性调频(LFM)信号,其他宽带信号形式如非线性调频、相位编码信号(如巴克码、P码)、频率步进信号等也可用于高分辨率成像,但LFM因其实现简单、易于产生大时宽带宽积而被广泛使用。一个典型的LFM发射信号 $s_t(t)$ 可以表示为:
\begin{equation}
    s_t(t) = \text{rect}\left(\frac{t}{T_p}\right) A_t \exp\left(j 2\pi f_c t + j \pi \gamma t^2\right)
    \label{eq:lfm_signal}
\end{equation}
其中,$t$ 表示快时间(fast time),$T_p$ 是脉冲持续时间,$A_t$ 是发射信号幅度,$f_c$ 是载波中心频率,$\gamma = B / T_p$ 是调频斜率。$\text{rect}(u)$ 是矩形窗函数。信号的瞬时频率为 $f(t) = f_c + \gamma t$,$t \in [-T_p/2, T_p/2]$,覆盖的带宽为 $B = |\gamma| T_p$。

假设目标可以由 $P$ 个理想的点散射中心模型近似描述。设第 $i$ 个散射中心的位置向量为 $\mathbf{r}_i$,其相对于雷达的初始距离为 $R_i = ||\mathbf{r}_i||$,对应的雷达散射系数为 $\sigma_i$(与频率、角度、极化相关)。在远场假设下,并考虑信号传播路径损耗,当雷达发射信号 $s_t(t)$ 后,经过目标散射并在雷达处接收到的回波信号 $s_r(t)$ 可以表示为所有散射中心回波的叠加:
\begin{equation}
    s_r(t) \approx \sum_{i=1}^{P} A_r \frac{\sigma_i}{R_i^2} s_t\left(t - \tau_i(t)\right)
    \label{eq:received_signal_sum_amplitude}
\end{equation}
其中,$A_r$ 是与雷达系统参数(如天线增益、发射功率)相关的幅度因子,$\tau_i(t) = 2 R_i(t) / c$ 是第 $i$ 个散射中心的瞬时双程传播时延,$R_i(t)$ 是 $t$ 时刻该散射中心到雷达的瞬时距离,$c$ 是光速。

在单个脉冲持续时间 $T_p$ 内,对于非机动或慢速目标,通常采用“冻结目标”或“走停”近似。在此近似下,$R_i(t) \approx R_i$。将回波信号下变频到基带,滤除高频项,并补偿固定的路径损耗和系统增益后,基带接收信号 $s_{r,base}(t)$ 近似为:
\begin{equation}
    s_{r,base}(t) \approx \sum_{i=1}^{P} \sigma'_i \text{rect}\left(\frac{t-\tau_i}{T_p}\right) \exp\left(j \pi \gamma (t-\tau_i)^2\right)
    \label{eq:received_baseband}
\end{equation}
其中 $\sigma'_i$ 是包含了幅度、散射相位以及传播相位 $\exp(-j 2\pi f_c \tau_i)$ 的等效复散射系数,$\tau_i = 2R_i/c$。

为了从接收信号中获得高距离分辨率,需要进行脉冲压缩处理,这通常通过与发射信号的复共轭进行匹配滤波来实现。匹配滤波器的冲激响应(基带形式)为 $h(t) = s_t^*(-t) \exp(-j 2\pi f_c t)$(或其归一化版本)。匹配滤波器的输出 $s_o(t)$ 是输入信号与滤波器冲激响应的卷积:
\begin{equation}
    s_o(t) = s_{r,base}(t) * h(t) = \int_{-\infty}^{\infty} s_{r,base}(\tau) h(t-\tau) d\tau
    \label{eq:matched_filtering}
\end{equation}
对于理想的LFM信号和单个点目标,匹配滤波输出近似为一个峰值位于 $t = \tau_1$ 的sinc函数,其幅度包络为 $|\text{sinc}(B(t - \tau_1))|$。对于由多个散射中心组成的目标,在忽略散射中心之间的相互耦合以及满足分辨率要求的前提下,匹配滤波的输出近似为各个散射中心响应的相干叠加:
\begin{equation}
    s_o(t) \approx \sum_{i=1}^{P} \sigma''_i \text{sinc}\left(B(t - \tau_i)\right) \exp(-j 2\pi f_c \tau_i)
    \label{eq:pulse_compression_output_phase}
\end{equation}
其中 $\sigma''_i$ 是包含了原始散射幅度、脉冲压缩增益等因素的复幅度系数。此处的相位项 $\exp(-j 2\pi f_c \tau_i)$ 非常重要,它导致了不同散射中心响应之间的相干干涉。

高分辨率距离像(HRRP)通常定义为脉冲压缩后输出信号的幅度(或功率)沿距离轴的分布。令距离 $r = ct/2$,则HRRP函数 $p(r)$ 可以表示为:
\begin{equation}
    p(r) = |s_o(t)|_{t=2r/c} \approx \left| \sum_{i=1}^{P} \sigma''_i \text{sinc}\left(\frac{2B}{c}(r - R_i)\right) \exp(-j \frac{4\pi f_c R_i}{c}) \right|
    \label{eq:hrrp_definition_complex}
\end{equation}
其中 $R_i = c\tau_i/2$ 是第 $i$ 个散射中心沿雷达视线的投影距离。该式更清晰地表明,HRRP是目标各散射中心响应(具有幅度和相位)在距离轴上相干叠加后的幅度包络。其能够分辨两个散射中心的最小距离间隔,即距离分辨率 $\Delta R$,由信号带宽 $B$ 决定:
\begin{equation}
    \Delta R = \frac{\kappa c}{2B}
    \label{eq:range_resolution_kappa}
\end{equation}
其中 $\kappa$ 是与窗函数和分辨率定义相关的因子,对于矩形窗和瑞利准则(主瓣峰值到第一零点),$\kappa \approx 1$。实际应用中,为抑制旁瓣,常使用非矩形窗函数(如汉宁窗、海明窗),这会略微展宽主瓣,即 $\kappa > 1$。

然而,式~(\ref{eq:hrrp_definition_complex})描述的是理想情况下的HRRP。在实际雷达系统中,接收到的信号总是伴随着噪声和杂波。设 $n(t)$ 表示基带接收机噪声和可能的外部干扰,通常建模为零均值加性高斯白噪声(Additive White Gaussian Noise, AWGN),其功率谱密度为 $N_0$。设 $c(t)$ 表示来自环境背景(地面、海面、云雨等)的杂波回波。则实际接收到的基带信号应为:
\begin{equation}
    s_{r,base}^{noisy}(t) = s_{r,base}(t) + c(t) + n(t)
    \label{eq:received_noisy}
\end{equation}
经过匹配滤波后,输出信号变为:
\begin{equation}
    s_o^{noisy}(t) = s_o(t) + c_o(t) + n_o(t)
    \label{eq:output_noisy}
\end{equation}
其中 $s_o(t)$ 是理想目标回波的脉压输出(式~(\ref{eq:pulse_compression_output_phase})),$c_o(t) = c(t) * h(t)$ 是杂波的脉压输出,$n_o(t) = n(t) * h(t)$ 是噪声的脉压输出。如果 $n(t)$ 是AWGN,则 $n_o(t)$ 也是零均值高斯过程,但不再是白噪声,其自相关函数由 $h(t)$ 决定。杂波 $c(t)$ 的统计特性通常更复杂,可能是非高斯的、相关的,并且其强度可能随距离、角度、环境类型(如不同地貌、海况)变化,常用的模型有瑞利分布(大量独立小散射体)、韦伯分布、对数正态分布或K分布(用于描述海杂波或地杂波的拖尾现象)~\cite{X}。

因此,实际观测到的HRRP样本 $p^{noisy}(r)$ 是含有噪声和杂波的脉压输出的幅度:
\begin{equation}
    p^{noisy}(r) = |s_o^{noisy}(t)|_{t=2r/c} = |s_o(t) + c_o(t) + n_o(t)|_{t=2r/c}
    \label{eq:hrrp_noisy}
\end{equation}
噪声和杂波的存在会污染甚至淹没目标信号 $s_o(t)$,导致HRRP形态失真、细节丢失、出现虚假峰值,从而严重影响识别性能。信噪比(Signal-to-Noise Ratio, SNR)或信杂比(Signal-to-Clutter Ratio, SCR)是衡量信号质量的重要指标。例如,脉压后的峰值信噪比可以定义为 $\text{SNR}_{peak} = \max |s_o(t)|^2 / E[|n_o(t)|^2]$。低SNR/SCR是RATR(特别是小样本RATR)面临的第一个严峻问题。后续章节需要研究的算法应具备在低SNR/SCR下(即面对式~(\ref{eq:hrrp_noisy})形式的输入时)的鲁棒性。

% --- 示意图占位符 ---
\begin{figure}[h!]
    \centering
    % \includegraphics[width=0.8\linewidth]{figures/hrrp_formation_noise.pdf} % 可能需要新的图
    \fbox{图 2.1:HRRP成像原理及噪声影响示意图 (占位符)}
    \caption{在图2.1基础上增加噪声和杂波对接收信号和最终HRRP形态影响的示意。}
    \label{fig:hrrp_formation_noise} % 更新标签
\end{figure}

现在我们来更形式化地描述HRRP的角度敏感性问题。如前所述,HRRP的形态依赖于目标姿态角 $(\theta, \phi)$。我们可以将理想HRRP函数显式地写为 $p(r; \theta, \phi)$,或者对于离散HRRP向量,记为 $\mathbf{p}(\theta, \phi) \in \mathbb{R}^N$。根据式~(\ref{eq:hrrp_definition_complex}),这种依赖性主要来源于投影距离 $R_i(\theta, \phi) = \mathbf{r}_i \cdot \hat{\mathbf{k}}(\theta, \phi)$ 和复幅度 $\sigma''_i(\theta, \phi)$ 对视线向量 $\hat{\mathbf{k}}(\theta, \phi)$ 的依赖。当姿态角发生一个微小的变化 $(\Delta\theta, \Delta\phi)$ 时,视线向量变化 $\Delta\hat{\mathbf{k}}$,导致投影距离变化 $\Delta R_i = \mathbf{r}_i \cdot \Delta\hat{\mathbf{k}}$,幅度 $\sigma''_i$ 也可能发生变化 $\Delta\sigma''_i$。由于HRRP是相干叠加的结果,即使 $\Delta R_i$ 很小(小于 $\Delta R$),但其引起的相位变化 $\Delta\psi_i = - \frac{4\pi f_c}{c} \Delta R_i$ 可能很大(因为 $f_c \gg B$),导致不同散射中心响应的干涉状态(相长或相消)发生剧烈改变,从而引起HRRP幅度 $p(r; \theta, \phi)$ 的快速振荡。我们可以用HRRP向量之间的距离(如欧氏距离)来衡量角度敏感性。对于同一目标 $y$,其在两个不同姿态角 $(\theta_1, \phi_1)$ 和 $(\theta_2, \phi_2)$ 下的HRRP样本 $\mathbf{p}_1 = \mathbf{p}(\theta_1, \phi_1)$ 和 $\mathbf{p}_2 = \mathbf{p}(\theta_2, \phi_2)$ 之间的距离 $d(\mathbf{p}_1, \mathbf{p}_2)$ 可能非常大,即使角度差 $\sqrt{(\theta_1-\theta_2)^2 + (\phi_1-\phi_2)^2}$ 很小。同时,可能存在另一个不同目标 $y'$ 在某个姿态角 $(\theta_3, \phi_3)$ 下的HRRP样本 $\mathbf{p}_3$ 与 $\mathbf{p}_1$ 非常接近,即 $d(\mathbf{p}_1, \mathbf{p}_3)$ 很小。这种现象即 $d(\mathbf{p}_1, \mathbf{p}_2) \gg d(\mathbf{p}_1, \mathbf{p}_3)$,其中 $y_1=y_2=y, y_3=y', y \neq y'$,严重违反了许多模式识别算法所依赖的“类内距离小、类间距离大”的假设。这就是HRRP角度敏感性对识别带来的核心困难,也是小样本RATR面临的第二个关键问题。后续章节需要设计的算法必须能够处理或减轻这种极端角度敏感性的影响。

\subsection{典型空天目标电磁散射特性}
\label{subsec:scattering_characteristics}

理解空天目标的电磁散射特性是深入分析HRRP数据并设计有效识别算法的基础。目标的散射特性决定了雷达接收到的回波信号的强度、相位、极化等信息,从而决定了HRRP的形态。

目标的雷达散射截面积(Radar Cross Section,RCS),记为 $\sigma$,是定量描述目标在特定方向上散射雷达波能力强弱的关键物理量。其严格定义为:
\begin{equation}
    \sigma = \lim_{R \to \infty} 4\pi R^2 \frac{P_s}{P_i} = \lim_{R \to \infty} 4\pi R^2 \frac{|\mathbf{E}_s|^2}{|\mathbf{E}_i|^2}
    \label{eq:rcs_definition}
\end{equation}
其中,$R$ 是距离,$P_i$ 和 $P_s$ 分别是入射和散射功率密度,$\mathbf{E}_i$ 和 $\mathbf{E}_s$ 分别是入射和散射电场强度。RCS具有面积的量纲,单位通常是平方米(m²)或分贝平方米(dBsm)。对于一个复杂目标,RCS不仅依赖于目标的物理属性(尺寸、形状、材料),还强烈地依赖于雷达的工作参数(频率 $f_c$、极化方式 $\text{pol}$)和观测几何(入射波方向 $(\theta_i, \phi_i)$ 和散射波方向 $(\theta_s, \phi_s)$)。对于单站雷达,入射和散射方向相同,RCS通常表示为 $\sigma(f_c, \text{pol}, \theta, \phi)$,其中 $(\theta, \phi)$ 是目标本体坐标系下的姿态角。

根据电磁场理论,在高频区(目标尺寸远大于波长 $\lambda = c/f_c$),目标的总散射场可以看作是由目标表面感应电流和感应磁流辐射产生的场,其贡献主要来自于局部区域,可以分解为几种基本的散射机制~\cite{X}。镜面反射发生在尺寸远大于波长的光滑曲面上,满足几何光学反射定律,散射能量集中在镜面反射方向,通常形成RCS方向图中的强峰值。边缘绕射发生在目标几何形状的突变处,如机翼、尾翼的边缘,舵面、舱门的缝隙边缘等。根据几何绕射理论(Geometrical Theory of Diffraction,GTD),边缘绕射的能量相对较弱,但方向性比镜面反射弱,可在更宽的角度范围内观测到。尖顶或角点绕射发生在目标的尖顶(如机头、导弹弹头)或角点处。爬行波是电磁波入射到目标光滑曲面的阴影边界时,激发沿表面传播的波,并在目标的背向区域再次辐射,对低频散射和阴影区散射有重要贡献。行波是在细长结构(如机翼前缘、天线臂)上,入射波可能激发沿结构传播的波,并在端点或不连续处产生辐射。腔体散射对于具有开放式腔体结构的目标(如飞机的发动机进气道、尾喷口,座舱等)非常重要,入射电磁波会进入腔体内部,经过多次反射和模式转换后再辐射出来,可能形成非常强的散射,并且其散射特性对频率和观测角度通常极为敏感~\cite{X}。一个复杂空天目标(如飞机)的RCS随姿态角的变化曲线通常呈现出极其复杂的起伏形态,峰谷差异可达数十dB,并且在很小的角度间隔内就可能发生剧烈变化~\cite{X}。这是因为随着姿态角的改变,雷达视线照射到目标的不同部位,主导的散射机制会发生转换,同时来自不同散射源(散射中心)的贡献之间的相干干涉关系也会随之改变,导致总散射场的强度发生快速振荡。

为了在高频区对目标的散射特性进行建模和分析,散射中心模型被广泛应用~\cite{X}。该模型将目标的总散射场近似为来自目标上有限个离散的等效散射中心的贡献的相干叠加。属性散射中心模型(Attributed Scattering Center Model,ASCM)是其中一种常用模型,它不仅给出了散射中心的位置,还描述了其散射强度随频率和角度变化的特性。根据ASCM,第 $i$ 个散射中心对总散射场 $E_s$ 的贡献 $E_i$ 可以表示为频率 $f$ 和视线向量 $\hat{\mathbf{k}}$ 的函数:
\begin{equation}
    E_i(f, \hat{\mathbf{k}}) \approx A_i(\text{pol}) \left(\frac{j f}{f_{ref}}\right)^{\alpha_i} S_i(f, \hat{\mathbf{k}}) \exp\left(-j \frac{4\pi f}{c} \mathbf{r}_i \cdot \hat{\mathbf{k}}\right)
    \label{eq:ascm}
\end{equation}
在此式中,$A_i(\text{pol})$ 是第 $i$ 个散射中心在参考频率 $f_{ref}$ 和特定极化下的复幅度。$\alpha_i$ 是频率依赖因子,描述了散射幅度随频率的变化关系,其取值与散射机制类型有关(例如,$\alpha_i=1$ 对应镜面反射,$\alpha_i=0$ 对应边缘绕射)。$S_i(f, \hat{\mathbf{k}})$ 是角度依赖因子,描述散射强度随观测角度的变化。对于局部化散射中心,$S_i \approx 1$;对于分布式散射中心(如长直边缘),$S_i$ 可能具有 $\text{sinc}$ 函数形式。$\mathbf{r}_i$ 是第 $i$ 个散射中心的位置向量。$\exp(-j \frac{4\pi f}{c} \mathbf{r}_i \cdot \hat{\mathbf{k}})$ 是由位置决定的相位项,其中 $\mathbf{r}_i \cdot \hat{\mathbf{k}} = R_i$ 即为该散射中心沿视线的投影距离。将式~(\ref{eq:ascm})代入宽带回波模型并进行脉冲压缩,即可得到基于散射中心模型的HRRP表达式。可以看出,HRRP的形态直接由各散射中心的位置 $\mathbf{r}_i$、类型 $\alpha_i$、强度 $A_i$ 以及它们随频率 $f$ 和视线 $\hat{\mathbf{k}}$ 的变化规律 $S_i$ 共同决定。当姿态角 $(\theta, \phi)$ 变化时,$\hat{\mathbf{k}}$ 改变,导致相位项中的投影距离 $R_i$、角度依赖因子 $S_i$ 以及可能的幅度 $A_i$ 都发生变化,进而造成HRRP形态的剧烈改变。这进一步从物理模型层面印证了HRRP的角度敏感性。

% --- 示意图占位符 ---
\begin{figure}[h!]
    \centering
    % \includegraphics[width=0.7\linewidth]{figures/rcs_angle.pdf}
    \fbox{图 2.2: 典型飞机RCS随方位角变化曲线 (占位符)}
    \caption{示意性展示一个典型飞机模型在固定俯仰角下,其RCS随方位角变化的剧烈起伏特性。}
    \label{fig:rcs_angle}
\end{figure}

此外,目标的微动,例如飞机涡轮发动机叶片的旋转、直升机旋翼的转动、导弹弹体的进动或章动等,会在雷达回波中引入除主体平动多普勒之外的附加频率调制,产生微多普勒效应~\cite{X}。微多普勒特征包含了目标的精细结构和运动状态信息,是目标识别的另一重要信息来源,尤其对于区分结构相似但内部运动部件不同的目标具有潜力。但微多普勒分析通常需要对信号进行时频分析,与主要基于HRRP的识别方法有所不同,本论文后续章节暂不深入探讨微多普勒特征的利用。

综上所述,空天目标复杂的几何结构和材料构成导致了其独特的电磁散射特性,这种特性对频率和观测角度高度敏感,是形成复杂多变HRRP数据的物理根源。深刻理解这些散射机理和特性,对于后续章节中设计能够适应HRRP数据复杂性(特别是角度敏感性)和环境干扰(如噪声)的鲁棒识别算法至关重要。

\section{基于深度学习的 RATR 技术}
\label{sec:深度学习_ratr}

近年来,深度学习凭借其强大的特征学习和非线性建模能力,已成为推动RATR技术发展的主要动力。基于深度学习的RATR方法旨在通过构建深度神经网络模型,自动从雷达数据中学习具有判别力的特征表示,实现端到端的识别。

\subsection{RATR 的深度学习框架}
\label{subsec:深度学习_framework}

基于深度学习的RATR系统通常遵循一个标准的监督学习框架。假设拥有训练数据集 $D_{train} = \{(\mathbf{x}_i, y_i)\}_{i=1}^{M}$,其中 $\mathbf{x}_i \in \mathcal{X}$ 是雷达观测数据(如HRRP向量 $\mathbf{p}_i \in \mathbb{R}^N$),$y_i \in \mathcal{Y} = \{1, \dots, C\}$ 是对应的目标类别标签。目标是学习一个映射函数 $f_\Theta: \mathcal{X} \rightarrow \mathcal{Y}$,该函数由参数 $\Theta$ 控制,能对未知样本 $\mathbf{x}$ 预测其类别 $\hat{y} = f_\Theta(\mathbf{x})$。

在深度学习框架下,$f_\Theta$ 通常由一个DNN实现,可视为复合函数,由多个层组成,每层对其输入进行线性变换和非线性激活。一个包含 $L$ 层的前馈神经网络可表示为:
\begin{align}
    \mathbf{a}^{(l)} &= W^{(l)} \mathbf{h}^{(l-1)} + \mathbf{b}^{(l)} \label{eq:dnn_linear} \\
    \mathbf{h}^{(l)} &= \sigma^{(l)}(\mathbf{a}^{(l)}) \label{eq:dnn_activation}
\end{align}
其中,$l=1, \dots, L$ 是层索引,$\mathbf{h}^{(l)} \in \mathbb{R}^{d_l}$ 是第 $l$ 层的输出,$\mathbf{h}^{(0)} = \mathbf{x}$ 是输入。$W^{(l)} \in \mathbb{R}^{d_l \times d_{l-1}}$ 和 $\mathbf{b}^{(l)} \in \mathbb{R}^{d_l}$ 是权重和偏置。$\sigma^{(l)}(\cdot)$ 是非线性激活函数(如ReLU)。整个网络的参数为 $\Theta = \{W^{(l)}, \mathbf{b}^{(l)}\}_{l=1}^{L}$。

对于 $C$ 类分类任务,最后一层输出 $C$ 维的 logits $\mathbf{z} = \mathbf{a}^{(L)}$。通过 Softmax 函数将 logits 转换为概率分布 $\mathbf{p}(\mathbf{x}) = [p(y=1|\mathbf{x}), \dots, p(y=C|\mathbf{x})]^T$:
\begin{equation}
    p(y=c|\mathbf{x}; \Theta) = \frac{\exp(z_c)}{\sum_{c'=1}^{C} \exp(z_{c'})}, \quad c=1, \dots, C
    \label{eq:softmax}
\end{equation}
最终预测标签 $\hat{y}$ 选择概率最高的类别:
\begin{equation}
    \hat{y} = f_\Theta(\mathbf{x}) = \arg\max_{c \in \{1, \dots, C\}} p(y=c|\mathbf{x}; \Theta)
    \label{eq:prediction}
\end{equation}

模型训练的目标是找到最优参数 $\Theta^*$,使模型在训练数据 $D_{train}$ 上的预测尽可能准确,并具有良好的泛化能力。这通过最小化损失函数 $\mathcal{L}$ 实现。常用的是交叉熵损失:
\begin{equation}
    \mathcal{L}_{CE}(f_\Theta(\mathbf{x}_i), y_i) = -\sum_{c=1}^{C} \mathbb{I}(y_i=c) \log p(y=c|\mathbf{x}_i; \Theta) = -\log p(y=y_i|\mathbf{x}_i; \Theta)
    \label{eq:cross_entropy_loss}
\end{equation}
其中 $\mathbb{I}(\cdot)$ 是指示函数。训练目标是最小化训练集上的平均损失,通常加入正则化项 $\Omega(\Theta)$ 防止过拟合:
\begin{equation}
    \Theta^* = \arg\min_{\Theta} \left\{ \frac{1}{M} \sum_{i=1}^{M} \mathcal{L}_{CE}(f_\Theta(\mathbf{x}_i), y_i) + \lambda \Omega(\Theta) \right\}
    \label{eq:optimization_objective}
\end{equation}
其中 $\lambda \ge 0$ 是正则化系数。此优化问题通常采用基于梯度的迭代算法求解,如随机梯度下降(Stochastic Gradient Descent,SGD)及其变种~\cite{X}。利用反向传播算法计算梯度 $\nabla_\Theta \mathcal{L}$,并更新参数:
\begin{equation}
    \Theta_{t+1} \leftarrow \Theta_t - \eta_t \nabla_\Theta \mathcal{L}(\Theta_t; D_{batch})
    \label{eq:sgd_update}
\end{equation}
其中 $\eta_t$ 是学习率。

% --- 示意图占位符 ---
\begin{figure}[h!]
    \centering
    % \includegraphics[width=0.9\linewidth]{figures/深度学习_framework.pdf}
    \fbox{图 2.3: 基于深度学习的RATR框架示意图 (占位符)}
    \caption{展示一个典型的基于深度学习的RATR系统框架,包括输入雷达数据、深度神经网络、输出类别概率以及训练过程。}
    \label{fig:深度学习_framework}
\end{figure}

\subsection{典型深度学习模型及其适用性}
\label{subsec:typical_深度学习_models}

针对HRRP数据(一维序列、结构性、角度敏感性),研究人员已将多种深度学习模型架构成功应用于HRRP RATR任务。

一维卷积神经网络(1D-CNN)擅长通过局部连接、权值共享和池化操作提取数据中的局部结构特征,并具有一定的平移不变性。对于一维HRRP向量 $\mathbf{p} \in \mathbb{R}^N$,1D-CNN使用一组可学习的一维卷积核 $\mathbf{k}_j \in \mathbb{R}^F$ 在输入序列上滑动。第 $j$ 个卷积核产生的特征图 $\mathbf{h}_j$ 的第 $n$ 个元素计算如下:
\begin{equation}
    h_{j,n} = \sigma\left( \sum_{f=1}^{F} k_{j,f} p_{n+f-c} + b_j \right)
    \label{eq:1d_cnn_conv}
\end{equation}
其中 $\sigma(\cdot)$ 是激活函数,$b_j$ 是偏置。通过堆叠卷积层(可能使用不同核尺寸和数量,或引入空洞卷积扩大感受野)和池化层(如最大池化,用于降维和增强鲁棒性),1D-CNN能够自动学习HRRP中从低级的简单模式到高级的复杂结构特征的层次化表示。后续通常接全连接层(Fully Connected,FC)进行特征整合和最终分类~\cite{X}。CNN的结构,特别是其深度(层数)和宽度(通道数),以及残差连接(ResNet~\cite{X})、密集连接(DenseNet~\cite{X})等先进结构,都可以被适配到1D场景,以增强模型的表达能力和训练稳定性。1D-CNN特别适合提取HRRP中的局部距离相关性特征,如散射中心的形状、宽度及相对位置关系。

RNN与LSTM适合处理序列数据,例如,当HRRP数据以随时间 $t$ 或角度 $\alpha$ 变化的序列 $\mathbf{P} = (\mathbf{p}_1, \mathbf{p}_2, \dots, \mathbf{p}_T)$ 形式给出时,其中 $\mathbf{p}_t \in \mathbb{R}^N$ 是第 $t$ 时刻(或第 $t$ 个角度)的HRRP样本。RNN通过循环连接,将前一时刻的隐藏状态 $\mathbf{h}_{t-1}$ 与当前输入 $\mathbf{p}_t$ 结合,计算当前隐藏状态 $\mathbf{h}_t$:
\begin{equation}
    \mathbf{h}_t = \sigma_h(W_{hh} \mathbf{h}_{t-1} + W_{xh} \mathbf{p}_t + \mathbf{b}_h)
    \label{eq:rnn_recurrence}
\end{equation}
最终的序列表示(例如 $\mathbf{h}_T$ 或所有 $\mathbf{h}_t$ 的平均/最大池化)被用于分类。标准RNN在处理长序列时容易遇到梯度消失或爆炸问题。LSTM~\cite{X}通过引入遗忘门 $\mathbf{f}_t$、输入门 $\mathbf{i}_t$、输出门 $\mathbf{o}_t$ 和细胞状态 $\mathbf{C}_t$ 来解决此问题,其更新过程数学形式如下:
\begin{align}
    \mathbf{f}_t &= \sigma_g(W_f [\mathbf{h}_{t-1}, \mathbf{p}_t] + \mathbf{b}_f) \label{eq:lstm_f} \\
    \mathbf{i}_t &= \sigma_g(W_i [\mathbf{h}_{t-1}, \mathbf{p}_t] + \mathbf{b}_i) \label{eq:lstm_i} \\
    \tilde{\mathbf{C}}_t &= \sigma_c(W_C [\mathbf{h}_{t-1}, \mathbf{p}_t] + \mathbf{b}_C) \label{eq:lstm_c_tilde} \\
    \mathbf{C}_t &= \mathbf{f}_t \odot \mathbf{C}_{t-1} + \mathbf{i}_t \odot \tilde{\mathbf{C}}_t \label{eq:lstm_c} \\
    \mathbf{o}_t &= \sigma_g(W_o [\mathbf{h}_{t-1}, \mathbf{p}_t] + \mathbf{b}_o) \label{eq:lstm_o} \\
    \mathbf{h}_t &= \mathbf{o}_t \odot \sigma_h(\mathbf{C}_t) \label{eq:lstm_h}
\end{align}
其中 $\odot$ 表示逐元素乘积,$\sigma_g$ 通常是Sigmoid,$\sigma_c, \sigma_h$ 通常是Tanh。LSTM及其变种(如GRU, BiLSTM)非常适合建模HRRP随角度变化的动态特性或捕捉单个HRRP内部沿距离单元的顺序依赖关系。注意力机制(Attention Mechanism)~\cite{X}也可以与RNN/LSTM结合,使模型能够动态地关注序列中更重要的部分。近年来,完全基于自注意力机制的Transformer模型~\cite{X}也开始被探索用于处理序列数据,包括HRRP序列,其并行计算能力和捕捉长距离依赖的能力是其优势。

自编码器(Auto-Encoder,AE)是一种无监督学习模型,由编码器 $g_\phi: \mathcal{X} \rightarrow \mathcal{Z}$ 和解码器 $f_\psi: \mathcal{Z} \rightarrow \mathcal{X}$ 组成,$\mathcal{Z}$ 是低维隐空间。通过最小化重构误差 $\mathbb{E}[ ||\mathbf{x} - f_\psi(g_\phi(\mathbf{x}))||^2 ]$ 进行训练:
\begin{equation}
    \phi^*, \psi^* = \arg\min_{\phi, \psi} \mathbb{E}_{\mathbf{x} \sim p_{data}(\mathbf{x})} [ ||\mathbf{x} - f_\psi(g_\phi(\mathbf{x}))||^2 ]
    \label{eq:ae_objective}
\end{equation}
训练好的编码器 $g_{\phi^*}$ 可作为有效的非线性特征提取器。AE的变种,如稀疏AE、去噪AE~\cite{X}、变分AE等,在HRRP特征学习、数据降维、去噪或数据增广方面有应用潜力。例如,DAE通过学习从含噪输入重构干净输出,可以学习到对噪声更鲁棒的特征表示。

这些深度学习模型为HRRP RATR提供了强大的表示学习能力。然而,它们的成功通常依赖于大量的、覆盖目标各种状态的标注数据。当数据量严重不足时(即小样本场景),这些参数量巨大的模型极易过拟合,泛化能力差,这正是FSL和元学习需要解决的核心问题。

\section{小样本 RATR 问题建模}
\label{sec:fsl_modeling}

FSL旨在使机器学习模型能够像人类一样,从极少数的样本中学习识别新的概念。元学习是实现FSL的一种主流且有效的范式,其核心思想是“学会学习”。本节将对FSL问题进行形式化定义,介绍元学习框架,并结合HRRP特性阐述小样本条件下特征判别性不足和语义信息利用匮乏问题的形式化理解。

\subsection{小样本学习定义}
\label{subsec:fsl_definition}

FSL问题通常设定在一个与传统监督学习不同的场景中。假设存在两个类别集合:基类别 $C_{base}$ 和新类别 $C_{novel}$,它们之间没有交集,即 $C_{base} \cap C_{novel} = \emptyset$。我们拥有一个基础数据集 $D_{base} = \{(\mathbf{x}_i, y_i) | y_i \in C_{base}\}_{i=1}^{M_{base}}$,其中包含来自基类别的大量标注样本。目标是利用在 $D_{base}$ 上学习到的知识,使模型能够在面对来自新类别 $C_{novel}$ 的任务时表现良好,即使每个新类别 $c \in C_{novel}$ 只有极少数($K$个)标注样本可用。这种设定被称为 $K$-shot 学习,其中 $K$ 通常很小(如1或5)。同时,任务通常涉及从 $C_{novel}$ 中区分 $N$ 个类别,称为 $N$-way $K$-shot 问题。

为了有效地训练和评估能够解决FSL问题的模型,研究界广泛采用了基于任务或情节的训练范式~\cite{X}。该范式通过在训练阶段模拟测试时的小样本场景来进行。具体来说,训练过程不是在整个 $D_{base}$ 上一次性完成,而是通过从 $D_{base}$ 中反复采样生成大量模拟的小样本学习任务(Episodes或Tasks)。一个典型的 $N$-way $K$-shot 分类任务 $\mathcal{T}$ 的构建过程如下:首先,从 $C_{base}$ 中随机无放回地选择 $N$ 个类别,构成该任务的类别子集 $C_{\mathcal{T}}$。然后,对于 $C_{\mathcal{T}}$ 中的每一个类别 $c$,从 $D_{base}$ 中该类别的样本里随机选择 $K$ 个标注样本,构成该任务的支持集(Support Set) $S_{\mathcal{T}} = \{(\mathbf{x}_i^s, y_i^s)\}_{i=1}^{N \times K}$,其中 $y_i^s \in C_{\mathcal{T}}$。支持集的作用是提供给模型在该特定任务上进行学习或适应的少量信息。接着,对于这 $N$ 个类别 $C_{\mathcal{T}}$,再从 $D_{base}$ 中(确保与 $S_{\mathcal{T}}$ 中的样本不同)选择一批样本,构成该任务的查询集(Query Set) $Q_{\mathcal{T}} = \{(\mathbf{x}_j^q, y_j^q)\}_{j=1}^{N_q}$,其中 $y_j^q \in C_{\mathcal{T}}$。查询集用于评估模型在利用支持集 $S_{\mathcal{T}}$ 进行学习/适应后的性能。通常,每个类别的查询样本数量 $N_q/N$ 会大于 $K$。

模型的训练目标是在大量按某种分布 $p(\mathcal{T})$ 采样生成的任务 $\mathcal{T}$ 上进行优化,使其能够最小化在各个任务查询集上的期望损失。通过这种在大量不同的小样本任务上进行“演练”的训练方式,期望模型能够学习到一种通用的、跨任务的“元知识”或“学习策略”。拥有这种能力的模型,在元测试(Meta-Testing)阶段,当面对一个由来自新类别 $C_{novel}$ 构成的、同样是 $N$-way $K$-shot 设置的新任务 $\mathcal{T}_{novel}$ 时,就能够利用其支持集 $S_{novel}$ 快速适应,并对其查询集 $Q_{novel}$ 中的样本做出准确的预测。

在小样本HRRP RATR问题中,$\mathbf{x}$ 是HRRP样本(向量 $\mathbf{p}$ 或序列 $\mathbf{P}$),$y$ 是目标类别标签。$D_{base}$ 可能是包含若干常见目标类型(如飞机A、B、C)在多种姿态角、多种信噪比下的大量HRRP样本。$C_{novel}$ 则包含一些新的、稀有的目标类型(如飞机X、Y、Z),每种只有 $K$ 个标注样本。训练时模拟大量 $N$-way $K$-shot 任务,测试时则在由 $C_{novel}$ 构成的 $N$-way $K$-shot 任务上评估模型的泛化识别能力。

小样本学习的核心困难在于,当每个类别的样本数量 $K$ 极小时,标准监督学习训练的深度模型 $f_\Theta$ 往往无法学习到具有良好泛化性的特征表示。这尤其体现在特征判别性不足的问题上。理想情况下,模型学习到的特征提取器 $\phi_\theta$(例如DNN的倒数第二层输出)应将来自同一类别的样本(即使它们由于姿态、噪声等原因看起来差异很大)映射到特征空间中的紧凑区域,并将不同类别的区域分离开。然而,当 $K$ 很小时,模型从每个类别看到的样本非常有限,可能无法捕捉到该类别的本质、不变的特征,而更容易关注到样本特有的、偶然的细节或噪声。这导致学习到的嵌入空间中,同类样本可能分散得很开(类内方差大),而不同类别的样本(特别是物理特征相似的类别)可能相互混杂(类间距离小)。形式化地,假设 $\phi_\theta(\mathbf{x})$ 是样本 $\mathbf{x}$ 的嵌入向量,对于属于类别 $y$ 的样本,其类内散布矩阵 $S_W = \sum_{c=1}^C \sum_{\mathbf{x}: y=c} (\phi_\theta(\mathbf{x}) - \mu_c)(\phi_\theta(\mathbf{x}) - \mu_c)^T$($\mu_c$ 为类别 $c$ 的均值向量)的迹可能很大,而类间散布矩阵 $S_B = \sum_{c=1}^C N_c (\mu_c - \mu)(\mu_c - \mu)^T$($\mu$ 为总均值, $N_c$ 为类别 $c$ 样本数)的迹相对较小,导致 Fisher 判别准则 $J = \text{tr}(S_W^{-1} S_B)$ 很小,即可分性差。这是小样本RATR面临的第三个关键问题(特征判别性不足)的数学根源之一。

此外,标准的模型 $f_\Theta(\mathbf{x})$ 通常只接收物理观测数据 $\mathbf{x}$ 作为输入。然而,如第一章所述,目标的语义信息 $s$(如功能类别、型号家族等)在小样本或特征模糊时可能提供重要的补充判别线索。当前框架下,语义信息 $s$ 并未被利用,即模型是 $f_\Theta(\mathbf{x})$ 而非 $f_\Theta(\mathbf{x}, s)$。这种语义信息利用的匮乏,形式上表现为模型输入空间的局限性,是小样本RATR面临的第三个问题的另一方面。后续章节将探讨如何将语义信息 $s$ 有效地融入学习框架。

\subsection{基于元学习的小样本 RATR 框架}
\label{subsec:meta_learning_framework}

元学习为解决上述FSL问题提供了一个强大的理论框架,旨在通过在大量相关任务上的学习,让模型掌握一种能够快速适应新任务的通用学习能力或先验知识(元知识)。形式化地,元学习的目标是学习一个元学习器 $\mathcal{A}$,其自身可能包含一组元参数 $\Phi$。当给定新任务 $\mathcal{T}$ 的支持集 $S_{\mathcal{T}}$ 时,元学习器能利用 $S_{\mathcal{T}}$ 适应自身,对查询样本 $\mathbf{x}^q$ 做出预测 $\hat{y}^q = \mathcal{A}(\mathbf{x}^q | S_{\mathcal{T}}; \Phi)$。元学习的训练过程(Meta-Training)旨在找到最优元参数 $\Phi^*$,使得元学习器在任务分布 $p(\mathcal{T})$ 上的期望性能最优。这通常通过最小化在所有训练任务 $\mathcal{T}_i \sim p(\mathcal{T})$ 上的平均损失来实现:
\begin{equation}
    \Phi^* = \arg\min_{\Phi} \mathbb{E}_{\mathcal{T}_i=(S_i, Q_i) \sim p(\mathcal{T})} [\mathcal{L}_{\mathcal{T}_i}(\Phi)]
    \label{eq:meta_objective}
\end{equation}
其中,$\mathcal{L}_{\mathcal{T}_i}(\Phi)$ 是元学习器在任务 $\mathcal{T}_i$ 上的损失,通常定义为在给定支持集 $S_i$ 条件下,在查询集 $Q_i$ 上的平均损失:
\begin{equation}
    \mathcal{L}_{\mathcal{T}_i}(\Phi) = \frac{1}{N_q} \sum_{j=1}^{N_q} \mathcal{L}( \mathcal{A}(\mathbf{x}_j^q | S_i; \Phi), y_j^q )
    \label{eq:task_loss_meta}
\end{equation}
$\mathcal{L}(\cdot, \cdot)$ 是基学习任务的损失函数(如交叉熵)。元参数 $\Phi$ 的优化通常也采用基于梯度的优化方法。以下介绍两种主流的元学习范式:

基于度量学习的元学习方法,其核心是学习一个通用的嵌入函数 $\phi_\theta: \mathcal{X} \rightarrow \mathbb{R}^d$(参数 $\theta$ 即元参数 $\Phi$),将HRRP样本映射到嵌入空间,使得同类样本靠近,异类样本远离。对于新任务 $\mathcal{T}=(S, Q)$,识别过程为:计算支持集 $S$ 中每个类别 $n$ 的原型 $\mathbf{c}_n$(通常是该类支持样本嵌入向量的均值):
\begin{equation}
    \mathbf{c}_n = \frac{1}{K} \sum_{\{(\mathbf{x}_i^s, y_i^s) \in S \mid y_i^s=n\}} \phi_\theta(\mathbf{x}_i^s)
    \label{eq:prototype_calculation}
\end{equation}
然后将查询样本 $\mathbf{x}^q$ 映射为 $\phi_\theta(\mathbf{x}^q)$,并根据其与各原型 $\mathbf{c}_n$ 的距离 $d(\cdot, \cdot)$(如欧氏距离平方)进行分类,例如选择距离最近的原型对应的类别:
\begin{equation}
    \hat{y}^q = \arg\min_{n \in \{1, \dots, N\}} d(\phi_\theta(\mathbf{x}^q), \mathbf{c}_n)
    \label{eq:protonet_prediction_argmin}
\end{equation}
或者通过Softmax计算概率:
\begin{equation}
    p(y=n | \mathbf{x}^q, S; \theta) = \frac{\exp(-\gamma d(\phi_\theta(\mathbf{x}^q), \mathbf{c}_n))}{\sum_{n'=1}^{N} \exp(-\gamma d(\phi_\theta(\mathbf{x}^q), \mathbf{c}_{n'}))}
    \label{eq:protonet_prediction_softmax_gamma} % Added gamma
\end{equation}
其中 $\gamma$ 是一个可选的尺度参数。在元训练阶段,参数 $\theta$ 通过最小化在大量采样任务查询集上的负对数似然损失来学习:
\begin{equation}
    \theta^* = \arg\min_{\theta} \mathbb{E}_{\mathcal{T}_i \sim p(\mathcal{T})} \left[ \sum_{(\mathbf{x}_j^q, y_j^q) \in Q_i} -\log p(y=y_j^q | \mathbf{x}_j^q, S_i; \theta) \right]
    \label{eq:protonet_meta_objective}
\end{equation}
ProtoNet~\cite{X}是这类方法的典型代表。其他基于度量的方法还包括MN~\cite{X}(使用注意力机制计算查询与支持样本的相似度)和RelationNet~\cite{X}(使用一个神经网络来学习相似度度量)。这类方法的优点是简洁、高效。然而,标准的度量学习方法可能对噪声敏感(噪声影响嵌入向量和距离计算),并且难以处理HRRP的极端角度敏感性(同一目标不同角度样本在嵌入空间可能距离很远,破坏原型代表性)。后续章节将针对这些问题对基于度量学习的元学习框架进行改进。

% --- 示意图占位符 ---
\begin{figure}[h!]
    \centering
    % \includegraphics[width=0.8\linewidth]{figures/protonet.pdf}
    \fbox{图 2.4: ProtoNet工作原理示意图 (占位符)}
    \caption{展示ProtoNet的基本思想:通过共享的嵌入网络将支持集和查询集样本映射到嵌入空间,计算每个类的原型,然后根据查询样本与原型的距离进行分类。}
    \label{fig:protonet}
\end{figure}

基于优化的元学习方法,其目标是学习一个模型的初始参数 $\theta_0$(作为元参数 $\Phi$),使得该初始参数能通过在新任务 $\mathcal{T}_i$ 的支持集 $S_i$ 上进行少量梯度下降更新,快速适应到对该任务最优的参数 $\theta_i'$。MAML~\cite{X}是该方向的代表作。其元训练包含两个嵌套优化循环。内循环是任务适应:对于任务 $\mathcal{T}_i=(S_i, Q_i)$,从当前元参数 $\theta$ 出发,使用 $S_i$ 计算损失 $\mathcal{L}_{S_i}(\theta)$,并进行一步或 $U$ 步梯度下降更新得到任务特定参数 $\theta_i'$。例如,一步更新:
\begin{equation}
    \theta_i'(\theta) = \theta - \alpha \nabla_\theta \mathcal{L}_{S_i}(\theta)
    \label{eq:maml_inner_update}
\end{equation}
其中 $\alpha$ 是内循环学习率。外循环是元优化:使用 $Q_i$ 评估适应后的模型 $f_{\theta_i'}$,计算损失 $\mathcal{L}_{Q_i}(\theta_i')$。元参数 $\theta$ 的更新基于所有任务的查询集损失梯度:
\begin{equation}
    \theta \leftarrow \theta - \beta \nabla_\theta \left( \sum_{\mathcal{T}_i \sim p(\mathcal{T})} \mathcal{L}_{Q_i}(\theta_i'(\theta)) \right)
    \label{eq:maml_outer_update}
\end{equation}
其中 $\beta$ 是外循环学习率。计算元梯度 $\nabla_\theta \mathcal{L}_{Q_i}(\theta_i'(\theta))$ 通常需要二阶导数,计算成本较高,存在一阶近似方法如FOMAML和Reptile~\cite{X}。MAML的核心是找到一个对任务损失变化敏感的初始化点。其优点是模型无关性。然而,MAML的训练可能不稳定,且对于HRRP的角度敏感性问题,仅仅几步梯度更新是否足以适应剧烈的特征变化也是一个疑问。

% --- 示意图占位符 ---
\begin{figure}[h!]
    \centering
    % \includegraphics[width=0.8\linewidth]{figures/maml.pdf}
    \fbox{图 2.5: MAML 工作原理示意图 (占位符)}
    \caption{示意性展示MAML的元训练过程:内循环使用支持集更新参数,外循环基于查询集损失更新元参数。}
    \label{fig:maml}
\end{figure}

元学习框架为解决小样本RATR问题提供了强大武器。通过元训练,模型有望学习到关于HRRP数据、类别关系及学习策略的元知识,从而在面对真实小样本场景时表现出更好的泛化性和适应能力。本论文后续章节将深入探讨如何将元学习框架(主要基于度量学习)与针对性的机制相结合,以应对噪声、角度敏感性及语义信息利用不足等具体问题。

\section{本章小结}
\label{sec:theory_summary}
本章首先从物理层面深入剖析了HRRP的成像机理,推导了宽带雷达信号模型与HRRP的数学表达式(式~(\ref{eq:hrrp_definition_complex})),揭示了其与目标散射中心分布的关系及距离分辨率特性(式~(\ref{eq:range_resolution_kappa}))。通过引入噪声与杂波模型(式~(\ref{eq:hrrp_noisy})),形式化了低SNR/SCR对HRRP识别构成的第一个关键问题。进一步地,结合电磁散射理论和散射中心模型(式~(\ref{eq:ascm}))分析,阐明了HRRP对姿态角 $p(r; \theta, \phi)$ 的极端敏感性源于散射投影变化和相干干涉,并指出这是识别面临的第二个关键问题。这些分析为后续章节理解HRRP数据特性、应对噪声干扰与角度变化奠定了物理基础。

其次,本章回顾了基于深度学习的RATR框架,包括优化目标(式~(\ref{eq:optimization_objective}))和典型模型(如1D-CNN,式~(\ref{eq:1d_cnn_conv});LSTM,式~(\ref{eq:lstm_h}))。接着,对小样本学习(FSL)问题进行了形式化定义,引入了情节式训练范式,并从特征空间角度分析了小样本下特征判别性不足及语义信息 $s$ 缺失构成的第三个关键问题。最后,重点介绍了元学习框架(式~(\ref{eq:meta_objective}))及基于度量学习(如ProtoNet,式~(\ref{eq:prototype_calculation})-(\ref{eq:protonet_meta_objective}))和基于优化(如MAML,式~(\ref{eq:maml_inner_update}), (\ref{eq:maml_outer_update}))这两大主流范式的数学原理。本章通过梳理相关基础理论并形式化关键问题,为后续章节针对性地提出基于元学习的解决方案提供了统一的数学语言和坚实的理论铺垫。