\begin{resume}
   %评阅版论文隐去阶段性成果具体信息,保留此段文字:
	
  %该论文作者在学期间取得的阶段性成果(学术论文等)已满足我校博士学位评 阅相关要求。为避免阶段性成果信息对专家评价学位论文本身造成干扰,特将论文作者的阶段性成果信息隐去。
  
  \section*{1. 学术论文} % 发表的和录用的合在一起

  \begin{enumerate}[label={[\arabic*]},itemsep=0pt,parsep=0pt,labelindent=26pt,labelwidth=*,leftmargin=0pt,itemindent=*,align=left]
   %[label=\textbf{[\arabic*]},itemindent=*, align=left] %老版本缩进对齐
   
  %\addtolength{\itemsep}{-.36\baselineskip}%缩小条目之间的间距,下面类似
  \item \textbf{\underline{Chen L}}, Sun X, Pan Z, Wang Z, Su X, Liu Z and Hu P. Make HRRPs to Be Graphs for Efficient Target Recognition [J]. IET Electronics Letters, 2024, 60(22):e70088. (SCI已发表,影响因子:1.202,一作.)
  \item \textbf{\underline{Chen L}}, Hu P, Pan Z, Sun X and Wang Z. A Deep Learning-Based Target Radial Length Estimation Method through HRRP Sequence [C]. 2024 IEEE 12th Asia-Pacific Conference on Antennas and Propagation (APCAP 2024), 2024. (EI已发表,一作.)
  \item \textbf{\underline{Chen L}}, Hu P, Pan Z, Sun X, Wang Z. Advancing Few-shot HRRP Target Recognition with Meta-learning and Graph Neural Network [C]. 中国电子学会第一届空天信息技术大会(AITC 2024), 2024.(获评“空天之星”优秀报告,一作.)
  \item Fan H, \textbf{\underline{Chen L}}, Xu C, Zhou J, Dai Y and Hu P. Few-shot Human Motion Recognition through Multi-Aspect mmWave FMCW Radar Data [C]. 2025 IEEE 45th International Symposium on Geoscience and Remote Sensing (IGARSS 2025), 2025. (CCF-C/EI已录用,二作兼通讯.)
  \item Feng S, \textbf{\underline{Chen L}}, Su X, Liu Q and Hu P. Enhancing HRRP RATR Robustness to Incomplete Aspect Angles via Supervised Contrastive Learning [C]. 2025 IEEE 10th International Conference on Intelligent Computing and Signal Processing (ICSP 2025), 2025. (EI已录用,二作.)
  \item 潘之梁,户盼鹤,\textbf{\underline{陈凌峰}},刘振. 基于深度增强IST网络的ISAR稀疏成像 [J]. 海军航空大学学报, 2024, 39(5):603-614. (中文核心.)
  \item Pan Z, Hu P, \textbf{\underline{Chen L}}, Su X and Liu Z. An ISAR Cross-range Scaling Method Based on Track Information [C]. 2024 IEEE 14th International Conference on Microwave and Millimeter Wave Technology (ICMMT 2024), 2024.(EI已录用.)
  \item \textbf{\underline{Chen L}}, Pan Z, Liu Q and Hu P. HRRPGraphNet++: Dynamic Graph Neural Network with Meta-Learning for Few-shot HRRP Radar Target Recognition [J]. MDPI Remote Sensing. (SCI在审,一作.)
  \item \textbf{\underline{Chen L}}, Hu Pan, Liu Q and Liu Z. GAF-MLGNN: An Efficient Meta-Learning Framework for Few-shot HRRP RATR with GNN. IEEE Transactions on Signal and Information Processing over Networks [J], 2025. (SCI在审,一作.)
  \item \textbf{\underline{Chen L}}, Hu P and Liu Z. Seeing What Few-shot Learners See: Contrastive Cross-Class Attribution for Explainability [C]. (CCF-A/EI在审,一作.)
  \item Hu P, \textbf{\underline{Chen L}}, Zhang Z and Liu Z. Feature Fusion CGAN Based HRRP Denoising and Reconstruction Method [J]. Chinese Journal of Electronics, 2025. (SCI在审,大修意见,通讯.)
  \end{enumerate}

  \section*{2. 发明专利} % 有就写,没有就删除
  \begin{enumerate}[label={[\arabic*]},itemsep=0pt,parsep=0pt,labelindent=26pt,labelwidth=*,leftmargin=0pt,itemindent=*,align=left]
  %[label=\textbf{[\arabic*]},itemindent=*, align=left] %老版本缩进对齐
  %\addtolength{\itemsep}{-.36\baselineskip}%
  \item 户盼鹤,\textbf{\underline{陈凌峰}},潘之梁,苏晓龙,王泽昊,刘振. 基于HRRP序列的目标径向长度估计方法、装置、设备和介质:ZL202411235169.5 (专利号.)
  \item 
  \item 
  \item 
  \item 
  \item 一种基于深度展开IAA网络的频谱估计方法.
  \item 户盼鹤,\textbf{\underline{陈凌峰}},潘之梁,苏晓龙,王泽昊,刘振. 基于深度学习的HRRP 序列目标径向长度估计方法及装置:202411251160.3 (申请号.)
  \end{enumerate}

    \section*{3. 科研项目} % 有就写,没有就删除
  \begin{enumerate}[label={[\arabic*]},itemsep=0pt,parsep=0pt,labelindent=26pt,labelwidth=*,leftmargin=0pt,itemindent=*,align=left]
  %[label=\textbf{[\arabic*]},itemindent=*, align=left] %老版本缩进对齐
  %\addtolength{\itemsep}{-.36\baselineskip}%
  \item 
  \end{enumerate}
\end{resume}
