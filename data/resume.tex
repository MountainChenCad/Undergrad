\begin{resume}
   %评阅版论文隐去阶段性成果具体信息,保留此段文字:
	
  %该论文作者在学期间取得的阶段性成果(学术论文等)已满足我校博士学位评 阅相关要求。为避免阶段性成果信息对专家评价学位论文本身造成干扰,特将论文作者的阶段性成果信息隐去。
  
  \section*{1. 学术论文} % 发表的和录用的合在一起

  \begin{enumerate}[label={[\arabic*]},itemsep=0pt,parsep=0pt,labelindent=26pt,labelwidth=*,leftmargin=0pt,itemindent=*,align=left]
   %[label=\textbf{[\arabic*]},itemindent=*, align=left] %老版本缩进对齐
   
  %\addtolength{\itemsep}{-.36\baselineskip}%缩小条目之间的间距,下面类似
  \item \textbf{\underline{Chen L}}, Sun X, Pan Z, Wang Z, Su X, Liu Z and Hu P*. Make HRRPs to Be Graphs for Efficient Target Recognition [J]. Electronics Letters, 2024, 60(22):e70088. (SCI已发表,影响因子:1.202,一作.)
  \item \textbf{\underline{Chen L}}, Hu P*, Pan Z, Sun X and Wang Z. A Deep Learning-Based Target Radial Length Estimation Method through HRRP Sequence [C]. 2024 IEEE 12th Asia-Pacific Conference on Antennas and Propagation (APCAP 2024). 2024. (EI已发表,一作.)
  \item \textbf{\underline{Chen L}}, Hu P*, Pan Z, Sun X, Wang Z. Advancing Few-shot HRRP Target Recognition with Meta-learning and Graph Neural Network [C]. 中国电子学会第一届空天信息技术大会(AITC 2024). 2024.(获评“空天之星”优秀报告,一作.)
  \item \textbf{\underline{陈凌峰}},{户盼鹤*},潘之梁,孙潇,王泽昊. 基于多特征融合CGAN的HRRP去噪重构与识别方法 [C]. 中国电子学会第六届全国电子战大会(CCEW 2024). 2024.(获评“激励计划”优秀论文,一作.)
  \item Fan H, \textbf{\underline{Chen L*}}, Xu C, Zhou J, Dai Y and Hu P. Few-shot Human Motion Recognition through Multi-Aspect mmWave FMCW Radar Data [C]. 2025 IEEE 45th International Symposium on Geoscience and Remote Sensing (IGARSS 2025). 2025. (CCF-C、EI已录用,二作兼通讯.)
  \item Feng S, \textbf{\underline{Chen L}}, Su X*, Liu Q and Hu P. Enhancing HRRP RATR Robustness to Incomplete Aspect Angles via Supervised Contrastive Learning [C]. 2025 IEEE 7th International Conference on Communications, Information System and Computer Engineering (CISCE 2025). 2025. (EI已录用,二作.)
  \item 潘之梁,户盼鹤*,\textbf{\underline{陈凌峰}},刘振. 基于深度增强IST网络的ISAR稀疏成像 [J]. 海军航空大学学报,2024, 39(5):603-614. (中文核心已发表.)
  \item Pan Z, Hu P*, \textbf{\underline{Chen L}}, Su X and Liu Z. An ISAR Cross-range Scaling Method Based on Track Information [C]. 2024 IEEE 14th International Conference on Microwave and Millimeter Wave Technology (ICMMT 2024). 2024.(EI已录用.)
  \item \textbf{\underline{Chen L}}, Pan Z, Liu Q and Hu P*. HRRPGraphNet++: Dynamic Graph Neural Network with Meta-Learning for Few-shot HRRP Radar Target Recognition [J]. Remote Sensing. (SCI在审,一作.)
  \item \textbf{\underline{Chen L}}, Hu P*, Liu Q and Liu Z. GAF-MLGNN: An Efficient Meta-Learning Framework for Few-shot HRRP RATR with GNN. IEEE Transactions on Signal and Information Processing over Networks [J]. 2025. (SCI在审,一作.)
  \item \textbf{\underline{Chen L}}, Hu P* and Liu Z. Seeing What Few-shot Learners See: Contrastive Cross-Class Attribution for Explainability [C]. 2025 IEEE/CVF 20th International Conference on Computer Vision (ICCV 2025). 2025.(CCF-A、EI在审,一作.)
  % \item \textbf{\underline{Chen L}}, Liu Q, Hu P*, Su X and Liu Z. Synergistic Cross-Modal Adaptation for Few-shot HRRP Target Recognition [C]. 2026 AAAI Conference on Artificial Intelligence (AAAI 2026). 2026.(CCF-A、EI在审,一作.)
  \item Hu P, \textbf{\underline{Chen L*}}, Zhang Z and Liu Z. Feature Fusion CGAN Based HRRP Denoising and Reconstruction Method [J]. Chinese Journal of Electronics, 2025. (SCI在审,大修意见,通讯.)
  \item \textbf{\underline{陈凌峰}},潘之梁,孙潇,王泽昊,户盼鹤*. 基于超分辨的ISAR图像对抗识别方法 [C]. 中国电子学会第十六届全国雷达学术年会. 2024.(已发表,一作.)
  \item Pan Z, Hu P, \textbf{\underline{Chen L}} and Liu Z. Feature Saliency-Driven Scattering Center Model Simplification: A Redundancy Reduction Framework for ISAR Deception. IEEE Transactions on Aerospace and Electronic Systems [J]. 2025. (SCI在审.)
  \item 孙潇,潘之梁,\textbf{\underline{陈凌峰}},杨威,户盼鹤*. 基于深度展开ISTA网络的弱辐射源二维频谱联合检测方法 [C]. 中国电子学会第十六届全国雷达学术年会,2024.(已发表.)
  \end{enumerate}

  \section*{2. 发明专利} % 有就写,没有就删除
  \begin{enumerate}[label={[\arabic*]},itemsep=0pt,parsep=0pt,labelindent=26pt,labelwidth=*,leftmargin=0pt,itemindent=*,align=left]
  %[label=\textbf{[\arabic*]},itemindent=*, align=left] %老版本缩进对齐
  %\addtolength{\itemsep}{-.36\baselineskip}%
  \item 户盼鹤,\textbf{\underline{陈凌峰}},潘之梁,苏晓龙,王泽昊,刘振. 基于HRRP序列的目标径向长度估计方法、装置、设备和介质 [P]. 中国. 发明专利. ZL202411235169.5. 2024. (已授权.)
  \item 户盼鹤,\textbf{\underline{陈凌峰}},潘之梁,苏晓龙,刘振. 融合格拉姆角场的小样本HRRP识别方法 [P]. 中国. 发明专利. XXXXXXXXXXXXXX. 2024. (已受理.)
  \item 户盼鹤,\textbf{\underline{陈凌峰}},苏晓龙,潘之梁,刘振. 基于特征融合条件生成对抗网络的HRRP去噪方法 [P]. 中国. 发明专利. XXXXXXXXXXXXXX. 2024. (已受理.)
  \item 户盼鹤,\textbf{\underline{陈凌峰}},刘振,潘之梁,苏晓龙. 小样本HRRP雷达目标识别方法、装置、设备和介质 [P]. 中国. 发明专利. XXXXXXXXXXXXXX. 2024. (已受理.)
  \item 户盼鹤,\textbf{\underline{陈凌峰}},潘之梁,苏晓龙,王泽昊,刘振. 基于深度学习的HRRP 序列目标径向长度估计方法及装置 [P]. 中国. 发明专利. 202411251160.3. 2024. (已受理.)
  \item 户盼鹤,孙潇,潘之梁,苏晓龙,\textbf{\underline{陈凌峰}},刘振,杨威. 一种基于深度展开IAA网络的频谱估计方法 [P]. 中国. 发明专利. 202410410176.8. 2024. (已受理.)
  \end{enumerate}

    \section*{3. 科研项目} % 有就写,没有就删除
  \begin{enumerate}[label={[\arabic*]},itemsep=0pt,parsep=0pt,labelindent=26pt,labelwidth=*,leftmargin=0pt,itemindent=*,align=left]
  %[label=\textbf{[\arabic*]},itemindent=*, align=left] %老版本缩进对齐
  %\addtolength{\itemsep}{-.36\baselineskip}%
  \item 湖南省自然科学基金青年学生基础研究项目:“基于元学习的小样本空天目标识别方法研究”. (项目负责人,5万元,2024.06-2025.06,在研.)
  \item 国家级大学生创新训练项目:“病毒-人蛋白质深度图模型构建”. (项目负责人,2万元,2023.06-2024.12,已结题,结论优秀.)
  \end{enumerate}

    \section*{4. 学科竞赛} % 有就写,没有就删除
  \begin{enumerate}[label={[\arabic*]},itemsep=0pt,parsep=0pt,labelindent=26pt,labelwidth=*,leftmargin=0pt,itemindent=*,align=left]
  %[label=\textbf{[\arabic*]},itemindent=*, align=left] %老版本缩进对齐
  %\addtolength{\itemsep}{-.36\baselineskip}%
  \item 国际级金奖,第二十届iGEM国际遗传工程机器设计大赛全球总决赛金奖暨治疗赛道冠军提名. 美国麻省理工学院,2023.(A类,团队负责人.)
  \item 国际级金奖,第十九届iGEM国际遗传工程机器设计大赛全球总决赛金奖. 美国麻省理工学院,2022.(A类,团队负责人.)
  \item 国际级优秀奖,第三十五届韩素音国际翻译大赛优秀奖. 中国翻译协会,2023.(A类,个人.)
  \item 国家级特等奖,第六届中国大学生5分钟科研英语演讲竞赛全国特等奖. 中国学术英语研究会,2023.(B类,团队负责人.)
  \item 国家级一等奖,第十七届iCAN全国大学生创新创业大赛全国一等奖. 中国信息协会,2023.(A类,技术骨干.)
  \item 国家级二等奖,第十九届挑战杯大学生课外学术作品竞赛全国二等奖. 共青团中央,2024.(A类,团队负责人.)
  \item 国际级二等奖,第三十六届MCM国际大学生数学建模竞赛二等奖. 美国数学及其应用联合会,2023.(A类,团队负责人.)
  \item 国际级三等奖,第二十二届APMCM亚太大学生数学建模竞赛三等奖. 北京图像图形学学会,2021.(B类,团队负责人.)
  \item ...
  \item 省部级一等奖,第二十一届全国大学生数学建模竞赛湖南省一等奖. 中国工业与应用数学协会,2023.(B类,团队负责人.)
  \item 全军级二等奖,第二届“八一杯”全军军事英语能力竞赛全军二等奖. 军队院校英语教学联席会,2023.(B类,个人.)
  \item 省部级三等奖,2022“外研社·国才杯”大学生英语演讲比赛湖南省三等奖. 北京外国语大学,2022.(B类,个人.)
  \end{enumerate}
\end{resume}
